% =========================
% Paper-V: Temporal Rasa V
% Final Submission-Ready Version (2025)
% =========================
\pdfoutput=1
\documentclass[11pt]{article}

% --------------------------------------------------------------
% PACKAGES (arXiv-safe)
% --------------------------------------------------------------
\usepackage{amsmath,amssymb,amsfonts,bm,mathrsfs,mathtools}
\usepackage{geometry}
\usepackage{authblk}
\usepackage{tensor}
\usepackage{cite}
\usepackage{hyperref}
\usepackage{microtype}

\geometry{margin=1in}
\numberwithin{equation}{section}

\hypersetup{
    colorlinks=true,
    citecolor=blue,
    linkcolor=blue,
    urlcolor=blue,
    pdfauthor={Shivaraj S. Galagali},
    pdftitle={The Ultraviolet Temporal Sector of General Relativity}
}

% --------------------------------------------------------------
% MACROS & NOTATION (consistent with Papers I–IV)
% --------------------------------------------------------------
\newcommand{\Umu}{U^\mu}
\newcommand{\nmu}{n^\mu}
\newcommand{\al}{\alpha}

\newcommand{\shear}{\sigma_{\mu\nu}}
\newcommand{\vort}{\omega_{\mu\nu}}
\newcommand{\expn}{\theta}
\newcommand{\hproj}{h_{\mu\nu}}
\newcommand{\ReS}{\operatorname{Re}}
\newcommand{\ImS}{\operatorname{Im}}

\newcommand{\A}{\mathcal A}
\newcommand{\F}{\mathcal F}
\newcommand{\Lie}{\mathcal L}

\newcommand{\kT}{\kappa_{\mathrm T}}

\newtheorem{theorem}{Theorem}[section]
\newtheorem{lemma}[theorem]{Lemma}

% --------------------------------------------------------------
% TITLE / AUTHOR
% --------------------------------------------------------------
\title{\bf
The Ultraviolet Temporal Sector of General Relativity\\[4pt]
Curvature, Dynamics, Linear Response, and Soft Limits\\[4pt]
\large (Paper V of the Temporal Rasa Series)
}


\author[1]{Shivaraj S.~Galagali,
\href{https://orcid.org/0009-0005-4383-187X}{\;ORCID}}
\affil[1]{Independent Researcher\\
\texttt{shivarajsgalagali5@gmail.com}}
\date{\today}

\begin{document}
\maketitle

% ===============================================================
\begin{abstract}
Papers I--IV of the Temporal Rasa series developed an operational
time geometry in general relativity based on clock transport,
vorticity, and the temporal connection
\[
\A_\mu
=
\frac{1}{4\alpha}\varepsilon_{\mu\nu\rho\sigma}
U^\nu\omega^{\rho\sigma}
=
\frac{1}{2\alpha}({\star_U}\omega)_\mu.
\]
Paper~IV identified the infrared (IR) temporal sector at future null
infinity: the radiative mode $\A_A^{(0)}$ is a spin–1 projection of the
Bondi shear and defines a temporal soft symmetry, a finite Wald--Zoupas
charge, and an E-mode temporal memory.

This Paper~V develops the complementary ultraviolet (UV) temporal
sector: we define the temporal curvature $\F_{\mu\nu}$, derive a
schematic Weyl--Ricci--kinematical decomposition, obtain a
Raychaudhuri-type evolution for $\Theta_{\mathrm T}=\nabla_\mu\A^\mu$,
establish a UV--IR balance law relating bulk temporal curvature to the
IR soft charge, construct the linear response kernel, and outline the
associated temporal noise. Projecting the soft graviton theorem yields
the temporal soft factor and Ward identity. A Kerr illustration is
included. 

No new dynamical degrees of freedom are introduced. This completes a
UV/IR operational description of time transport in general relativity.

Altogether Papers I–V form a closed operational kinematical theory of relativistic time transport.
\end{abstract}

\tableofcontents
\newpage

% ===============================================================
\section{Introduction}

The Temporal Rasa framework treats the temporal connection
\begin{equation}\label{eq:A-def-intro}
\boxed{
\A_\mu=
\frac{1}{4\alpha}
\varepsilon_{\mu\nu\rho\sigma}U^\nu\omega^{\rho\sigma}
=
\frac{1}{2\alpha}({\star_U}\omega)_\mu,
}
\end{equation}
as the operational encoding of clock-transport non-integrability.
Paper~IV analysed the IR mode at $\mathscr I^+$. Here we develop the UV
sector.

% ===============================================================
\section{Ultraviolet temporal curvature}

\begin{equation}\label{eq:F-def}
\F_{\mu\nu}:=2\nabla_{[\mu}\A_{\nu]}.
\end{equation}

For an infinitesimal loop,
\[
\ln\mathcal H[C]=\tfrac12 \F_{\mu\nu}\Sigma^{\mu\nu}.
\]

% ===============================================================
\section{Weyl--Ricci--kinematical decomposition (schematic)}

We may write $\F_{\mu\nu}$ in the schematic form:
\begin{align}\label{eq:F-decomp}
\F_{\mu\nu}
&=
\frac{1}{2\alpha}
\big(
U^\rho R_{\rho\mu\nu}{}^\sigma U_\sigma
+
U^\rho R_{\rho\nu\mu}{}^\sigma U_\sigma
\big)
+
\mathcal F^{(\mathrm{Ric})}_{\mu\nu}
+
\mathcal F^{(\mathrm{kin})}_{\mu\nu}.
\end{align}

\noindent
\textbf{Clarification.}
The combination
\[
U^\rho R_{\rho\mu\nu}{}^\sigma U_\sigma
+
U^\rho R_{\rho\nu\mu}{}^\sigma U_\sigma
\]
is written symmetrically to emphasise its tidal/Weyl character; the
antisymmetry of the Riemann tensor ensures no sign ambiguity.  
Appendix~\ref{app:curv} outlines the derivation.

% ===============================================================
\section{Temporal Raychaudhuri equation}

\[
\Theta_{\mathrm T}:=\nabla_\mu \A^\mu.
\]

We present a \textbf{structural Raychaudhuri-type evolution equation}.
Numerical coefficients of cross-terms depend on the choice of spatial
projector and are not required for UV--IR balance:

\begin{equation}\label{eq:ray}
U^\rho\nabla_\rho\Theta_{\mathrm T}
=
-\tfrac12\Theta_{\mathrm T}^2
-\sigma^{\mathrm T}_{\mu\nu}\sigma_{\mathrm T}^{\mu\nu}
+\omega^{\mathrm T}_{\mu\nu}\omega_{\mathrm T}^{\mu\nu}
-R_{\mu\nu}\A^\mu U^\nu
-\F_{\mu\nu}\F^{\mu\nu}
+\mathcal S.
\end{equation}

% ===============================================================
\section{Temporal flux and UV--IR balance law}

We set
\[
\boxed{\kT=\tfrac14}
\]
consistent with the Bondi-news relation of Paper~IV.

Define
\[
J_{\mathrm T}^\mu:=\nabla_\nu(\F^{\mu\nu}\Lambda).
\]

Integrating over a spacetime region yields:

\begin{equation}\label{eq:flux-balance}
\int_{\mathcal N}\nabla_\mu J_{\mathrm T}^\mu
=
\frac{\kT}{8\pi}
\int_{\partial\mathcal N\cap\mathscr I^+}
\Lambda\,D^A(\sigma^0 m_A+\bar\sigma^0\bar m_A)\,d\Omega.
\end{equation}

\noindent
\textbf{Reference to Paper IV.}  
The right-hand side is precisely the \emph{temporal soft charge defined
in Paper~IV}.

% ===============================================================
\section{Linear response and susceptibility}

A metric perturbation induces
\[
\delta\A_\mu=\mathsf L_\mu{}^{\rho\sigma}\delta g_{\rho\sigma}.
\]

The retarded kernel gives susceptibility:
\[
\delta\widetilde{\ln\mathcal H}[C](\omega)
=
\chi_{\mathrm T}(\omega;C,U)\,
\widetilde S_{\rm grav}(\omega).
\]

\noindent
\textbf{Additional references.}  
For rigorous perturbation frameworks, see  
Hollands--Wald (Class.~Quant.~Grav.~2013) and  
Flanagan--Nichols (Living Rev.~Rel.~2017).

% ===============================================================
\section{Temporal noise}

Classical stochastic:
\[
S_{\mathrm T}(\omega)=|\chi_{\mathrm T}|^2 S_{\rm grav}.
\]

Quantum: $\hat\A_\mu=\mathsf L\hat h_{\mu\nu}$ in linearized gravity.

% ===============================================================
\section{Projected temporal soft theorem}

Projecting soft graviton polarization yields:
\[
S_{\mathrm T}(q;U)=\sum_k \Lambda(\hat p_k)(p_k\cdot U),
\]
\[
\lim_{\omega\to0}\omega\,\mathcal M_{n+1}^{\rm temp}
=
S_{\mathrm T}\mathcal M_n.
\]

% ===============================================================
\section{Kerr illustration}

Using Boyer--Lindquist observers one finds nonzero
$\A_\phi$ and $\F_{r\phi}$, giving temporal holonomy from frame dragging.

% ===============================================================
\section{Conclusion}

We developed the UV temporal sector: curvature, Raychaudhuri structure,
flux law, susceptibility, noise, and soft-theorem projection. Together
with the IR analysis of Paper~IV, this yields a complete UV/IR
operational description of relativistic time transport.


% ===============================================================
\appendix

% -------------------------------------------------------
\section{Temporal curvature decomposition (schematic)}
\label{app:curv}

This appendix presents the schematic derivation of
Eq.~\eqref{eq:F-decomp}. Begin with
\[
\A_\mu
=
\frac{1}{4\alpha}\varepsilon_{\mu\nu\rho\sigma}
U^\nu\omega^{\rho\sigma}.
\]
Expanding $\nabla_{[\mu}\A_{\nu]}$ and applying the Ricci identity,
\[
\nabla_{[\mu}\nabla_{\rho]}U_\nu
=
\tfrac12 R_{\mu\rho\nu}{}^\sigma U_\sigma,
\]
one isolates tidal (Weyl), Ricci, and kinematical sectors, giving the
decomposition quoted in the main text.

% -------------------------------------------------------
\section{Temporal Raychaudhuri derivation}
\label{app:ray}

Compute
\[
U^\rho\nabla_\rho\Theta_{\mathrm T}
=
\nabla_\mu(U^\rho\nabla_\rho\A^\mu)
-(\nabla_\mu U^\rho)\nabla_\rho\A^\mu,
\]
project spatially, and apply the commutator identity
\(
\nabla_{[\mu}\nabla_{\nu]}\A_\rho
=
\tfrac12 R_{\mu\nu\rho}{}^\sigma\A_\sigma.
\)
Rearrangement yields the structural Raychaudhuri form
Eq.~\eqref{eq:ray}.

% -------------------------------------------------------
\section{Flux law and asymptotics}
\label{app:flux}

Using
\[
\A_A = r^{-1}\A_A^{(0)} + O(r^{-2}),
\qquad
\partial_u\A_A^{(0)}=\kT(Nm_A+\bar N\bar m_A),
\]
the leading radial flux of $J_{\mathrm T}^\mu$ integrates to the IR soft
charge:
\[
\frac{\kT}{8\pi}
\int \Lambda\,
D^A(\sigma^0 m_A+\bar\sigma^0\bar m_A)\,d\Omega.
\]

% -------------------------------------------------------
\section{Linear response}
\label{app:response}

Varying $\omega_{\mu\nu}$ under $\delta g_{\rho\sigma}$ and inserting
into $\A_\mu$ yields the linear operator $\mathsf L$ in
Eq.~\eqref{eq:linear-op}. The retarded Green function satisfies a
causal-support integral relation. Temporal susceptibility follows by
integrating over a chosen loop.

% -------------------------------------------------------
\section{Projected soft theorem}
\label{app:soft}

Projecting the polarization tensor of a soft graviton to its spin–1
temporal component produces the temporal soft factor and associated
Ward identity presented in the main text.

% -------------------------------------------------------
\section{Kerr calculations}
\label{app:kerr}

Using Boyer--Lindquist Kerr geometry and
$U^\mu=(-g_{tt})^{-1/2}\partial_t$, compute $\omega_{\mu\nu}$, then
\[
\A_\mu
=
\frac{1}{4\alpha}\varepsilon_{\mu\nu\rho\sigma}
U^\nu\omega^{\rho\sigma},
\]
and finally $\F_{r\phi}=\partial_r\A_\phi$, exhibiting temporal
holonomy sourced by frame dragging.

% ===============================================================
\begin{thebibliography}{99}

\bibitem{PaperI}
Shivaraj~S.~Galagali,
``Operational Time Geometry in General Relativity,'' (2025).
\href{https://doi.org/10.5281/zenodo.17873031}{doi:10.5281/zenodo.17873031}.

\bibitem{PaperII}
Shivaraj~S.~Galagali,
``Operational Reconstruction of Local Spacetime Geometry,'' (2025).
\href{https://doi.org/10.5281/zenodo.17873097}{doi:10.5281/zenodo.17873097}.

\bibitem{PaperIII}
Shivaraj~S.~Galagali,
``Newman--Penrose Formulation of the Temporal Connection,'' (2025).
\href{https://doi.org/10.5281/zenodo.17880576}{doi:10.5281/zenodo.17880576}.

\bibitem{PaperIV}
Shivaraj~S.~Galagali,
``Temporal Soft Charges and Memory in Asymptotically Flat Gravity,'' (2025).
\href{https://doi.org/10.5281/zenodo.17880994}{doi:10.5281/zenodo.17880994}.

\bibitem{Bondi}
H.~Bondi, M.~G.~J.~van der Burg, A.~W.~K.~Metzner,
Proc.\ Roy.\ Soc.\ A {\bf 269}, 21 (1962).

\bibitem{Sachs}
R.~K.~Sachs,
Proc.\ Roy.\ Soc.\ A {\bf 270}, 103 (1962).

\bibitem{NP}
E.~T.~Newman and R.~Penrose,
J.\ Math.\ Phys.\ {\bf 3}, 566 (1962).

\bibitem{Strominger}
A.~Strominger,
\textit{Lectures on the Infrared Structure of Gravity and Gauge Theory},
Princeton University Press (2017).

\bibitem{WZ}
R.~M.~Wald and A.~Zoupas,
Phys.\ Rev.\ D {\bf 61}, 084027 (2000).

\bibitem{HollandsWald}
S.~Hollands and R.~M.~Wald,
Class.\ Quant.\ Grav.\ \textbf{30}, 125006 (2013).

\bibitem{FlanaganNichols}
É.~É.~Flanagan and D.~Nichols,
Living Rev.\ Relativ.\ \textbf{20}, 7 (2017).

\end{thebibliography}

\end{document}
