\pdfoutput=1
\documentclass[11pt]{article}

\usepackage{amsmath,amssymb,amsthm,bm,mathrsfs}
\usepackage{geometry}
\usepackage{tensor}
\usepackage{cite}
\usepackage{hyperref}
\usepackage{microtype}
\geometry{margin=1in}
\hypersetup{
  colorlinks=true,
  linkcolor=blue,
  citecolor=blue,
  urlcolor=blue,
  pdftitle={Classical Temporal Geometry in General Relativity},
  pdfauthor={Shivaraj S. Galagali}
}

% ----------------------------
% Macros
% ----------------------------
\newcommand{\A}{\mathcal A}
\newcommand{\F}{\mathcal F}
\newcommand{\Atemp}{\mathcal A^{(0)}}
\newcommand{\ThetaT}{\Theta_{\mathrm T}}
\newcommand{\al}{\alpha}
\newcommand{\shear}{\sigma}
\newcommand{\vort}{\omega}
\newcommand{\expn}{\theta}
\newcommand{\hproj}{h}
\newcommand{\Umu}{U^\mu}

% --------------------------------------------------------------
% TITLE & AUTHOR
% --------------------------------------------------------------
\title{\bf Classical Temporal Geometry in General Relativity}

\author{
Shivaraj S.~Galagali,{\href{https://orcid.org/0009-0005-4383-187X}{ORCID}}\\
\small Independent Researcher, Bengaluru, India\\
\small \texttt{shivarajsgalagali5@gmail.com}
}
\date{\today}

% --------------------------------------------------------------
\begin{document}
\maketitle

\begin{center}
{\Large \emph{Temporal Rasa Compendium: Part I}}\\[10pt]
\end{center}

\begin{abstract}
Temporal geometry in general relativity is operational: 
proper time depends on the path taken, and time standards transported
along different worldlines fail to remain synchronized in the presence of
vorticity.  This paper develops a unified formulation of this structure
using the \emph{temporal connection} 
\(
\A_\mu = (2\al)^{-1}\,\vort_{\nu\mu}U^\nu,
\)
whose holonomy encodes measurable clock desynchronisation.
Building upon the foundations established in Papers~I--V of the
Temporal Rasa series, we show that $\A_\mu$ admits a complete
constraint--evolution--flux--symmetry--entropy--canonical structure
analogous to the spin--2 BMS sector of asymptotically flat gravity.

We present twelve \emph{Temporal Laws}: a constraint equation, a
Raychaudhuri-type evolution, a UV--IR flux relation, a soft symmetry and
algebra, a canonical IR phase space, a modular Hamiltonian and entropy
law, a holographic dictionary, a Noether--Bianchi identity, and a
Hamilton--Jacobi principle.  These results demonstrate that temporal
geometry forms a closed structural subsystem of GR, derived entirely
from $(g_{\mu\nu},U^\mu)$ with no new propagating degrees of freedom.
\end{abstract}

\tableofcontents
\newpage

% --------------------------------------------
\section{Introduction}
\label{sec:intro}

Time in general relativity is not universal: different worldlines
accumulate different proper times, and transporting a local time standard
around a closed loop generically produces measurable desynchronisation.
This operational structure is encoded in the vorticity of an observer
congruence. Papers~I--V introduced the \emph{temporal connection}
\[
\A_\mu = \frac{1}{2\al}\,\vort_{\nu\mu}U^\nu,
\qquad 
\A_\mu U^\mu = 0,
\] \cite{PaperI,PaperIII}


and demonstrated that: 
(i) its holonomy quantifies clock desynchronisation \cite{PaperI};

(ii) its leading angular mode at $\mathscr I^+$ is the spin--1 projection
of the Bondi shear \cite{PaperIII};

(iii) the resulting IR temporal mode generates temporal soft symmetries,
charges and memory \cite{PaperIV}.


This paper consolidates and extends those results into a systematic
framework. The twelve laws presented here establish temporal geometry as
a closed structural subsystem of GR, with well-defined constraint,
evolution, conservation and holographic properties. 
They form Part~I of the Temporal Rasa Compendium.

\subsection*{Relation to the Temporal Rasa Series}

The twelve laws formulated in this Pillar summarise and unify results
established rigorously in Papers~I–V of the Temporal Rasa Series:
\begin{itemize}

    \item \textbf{Paper~I} \cite{PaperI} provides the operational definition of the temporal lapse 
    $\alpha$, the master evolution equation, and clock–transport holonomy.
    
    \item \textbf{Paper~II} \cite{PaperII} establishes the reconstruction of local geometry from 
    $(\theta, \sigma_{\mu\nu}, \omega_{\mu\nu})$, giving the structural
    foundation for Laws~2–5.
    
    \item \textbf{Paper~III} \cite{PaperIII} identifies the Newman--Penrose formulation of the 
    temporal connection and proves that its leading mode at $\mathscr I^+$ is 
    the spin--1 projection of the Bondi shear, underlying Laws~6–7.
    
    \item \textbf{Paper~IV} \cite{PaperIV} develops the infrared temporal sector: soft symmetry,
    Wald--Zoupas charge, and temporal memory, corresponding to Laws~8–10.
    
    \item \textbf{Paper~V} \cite{PaperV} introduces the ultraviolet temporal sector, temporal 
    curvature, Raychaudhuri structure, UV--IR balance law, and soft-theorem 
    projection, corresponding to Laws~11–12.

\end{itemize}


These papers collectively establish that temporal geometry introduces
\emph{no new fields and no new degrees of freedom}: every temporal
quantity used in this Pillar is a derived contraction or projection of
standard GR kinematics and Bondi radiative data.



% --------------------------------------------
\section{Preliminaries and Notation}
\label{sec:prelim}

We adopt the metric signature $(-,+,+,+)$ and set $c=G=1$.  
A timelike observer congruence $U^\mu$ satisfies $U^\mu U_\mu=-1$, and
its orthogonal projector is
\begin{equation}
\hproj_{\mu\nu} = g_{\mu\nu} + U_\mu U_\nu.
\end{equation}

The $1{+}3$ kinematical decomposition of $\nabla_\mu U_\nu$ is
\begin{equation}
\nabla_\mu U_\nu
=
\frac{1}{3}\,\expn\,\hproj_{\mu\nu}
+ \shear_{\mu\nu}
+ \vort_{\mu\nu}
- U_\mu a_\nu,
\end{equation}


where $\expn$ is expansion, $\shear$ shear,
$\vort$ vorticity, and $a_\nu = U^\mu\nabla_\mu U_\nu$ the
four-acceleration.

A reference congruence $n^\mu$ is used to define an operational lapse
\begin{equation}
\al = -U_\mu n^\mu >0.
\end{equation}

Near future null infinity $\mathscr I^+$ we use Bondi coordinates
$(u,r,x^A)$, with a null tetrad $(l^\mu,n^\mu,m^\mu,\bar m^\mu)$ and
Bondi shear $\sigma^0(u,x^A)$ and news $N=\partial_u\sigma^0$.

Throughout the paper, the temporal connection and all derived
quantities are understood as functionals of $(g_{\mu\nu},U^\mu)$ only.
There are no new degrees of freedom.

% --------------------------------------------
\section{Temporal Connection and Temporal Curvature}
\label{sec:temp-connection}

\subsection{Definition}

Given $U^\mu$ and $\al$, the \emph{temporal connection} is defined as
\begin{equation}
\A_\mu
=
\frac{1}{2\al}\,\vort_{\nu\mu}U^\nu,
\qquad
\A_\mu U^\mu = 0.
\label{eq:A-def}
\end{equation}
The holonomy around a small loop $C$ is
\begin{equation}
\oint_C \A_\mu dx^\mu
=
\frac{1}{2\al}\int_{\Sigma(C)} 
\vort_{\mu\nu}U^\nu\, d\Sigma^\mu,
\end{equation}
so $\A_\mu$ encodes operational clock desynchronisation.

\subsection{Temporal curvature}

The curvature of $\A_\mu$ is
\begin{equation}
\F_{\mu\nu} := 2\nabla_{[\mu}\A_{\nu]}.
\label{eq:F-def}
\end{equation}
For small surface elements $\Sigma^{\mu\nu}$,
\begin{equation}
\ln \mathcal{H}[C]
=
\frac12\,\F_{\mu\nu}\,\Sigma^{\mu\nu}
+O(\Sigma^2),
\end{equation}
so $\F_{\mu\nu}$ measures infinitesimal non-integrability of temporal
transport, analogous to curvature for spatial frames.

\subsection{Asymptotic projection and the IR temporal mode}

At large $r$, the angular components of $\A_\mu$ behave as
\begin{equation}
\A_A(u,r,x^A)
=
\frac{1}{r}\Atemp_A(u,x^A)
+ O(r^{-2}).
\end{equation}
Using the NP expansion of vorticity and shear (Paper~III), the leading
mode is
\begin{equation}
\Atemp_A
=
-\sigma^0\,\bar m_A
-\bar\sigma^0\,m_A,
\label{eq:A0-shear}
\end{equation}
the spin--1 projection of the Bondi shear.

Thus the IR temporal sector contains \emph{no new degrees of freedom};
it is fixed algebraically by $\sigma^0$.

\subsection{Relation to temporal memory}

From \eqref{eq:A0-shear},
\begin{equation}
\partial_u \Atemp_A
=
-N\,\bar m_A
-\bar N\,m_A,
\end{equation}
so the temporal memory between $u_i$ and $u_f$ is
\begin{equation}
\Delta \Atemp_A
=
-\!\int_{u_i}^{u_f}
\!\!\left( N\,\bar m_A + \bar N\,m_A \right) du,
\end{equation}
matching the E-mode part of displacement memory and verifying that
temporal memory is a repackaging of Bondi shear evolution.

\subsection{Interpretation}

\begin{quote}
The temporal connection is the observer-dependent but covariant object
encoding proper-time transport; its IR mode is simply a projection of
Bondi shear, so its entire structure is contained inside GR’s existing
degrees of freedom.  All “laws” derived below organise this structure
without introducing new fields.
\end{quote}

\newpage



% ================================================================
\section{Law I: Temporal Constraint Equation}
\label{sec:Law1_constraint}

\begin{quote}
\textbf{Law I (Temporal Constraint).}
On any spacelike or null hypersurface with induced metric $h_{\mu\nu}$  
and surface derivative $D_\mu$, the divergence of the spatial temporal  
connection satisfies a constraint of the form
\begin{equation}
D^A \A_A
=
\mathcal{S}_{\rm temp}
\bigl[
E_{\mu\nu},R_{\mu\nu};
(\expn,\shear,\vort,a_\mu),
\bigr],
\label{eq:temp-constraint}
\end{equation}
where $\mathcal{S}_{\rm temp}$ is a local scalar built from curvature  
and $1{+}3$ kinematical fields.  
No new degree of freedom enters: $D^A\A_A$ is fixed by $(g_{\mu\nu},U^\mu)$.
\end{quote}

% ------------------------------------------------------------
\subsection{Derivation sketch}

Insert the definition
\begin{equation}
\A_\mu = \frac{1}{2\alpha}\,\vort_{\nu\mu} U^\nu,
\end{equation}
into $D^A\A_A = h^{AB}D_A\A_B$, and expand using the Ricci identity  
\begin{equation}
\nabla_{[\mu}\nabla_{\nu]}U_\rho
=
\frac12 R_{\mu\nu\rho}{}^\sigma U_\sigma.
\end{equation}

Separating terms linear in curvature and quadratic in kinematical fields  
yields schematically
\begin{equation}
D^A\A_A
\simeq
\frac{1}{2\alpha}\,E_{\mu\nu}\,\shear^{\mu\nu}
+
\frac{1}{4\alpha}\,R_{\mu\nu}\,\vort^{\mu\nu}
+
D_\mu\!\left(\frac{1}{2\alpha}\,\vort^\mu{}_{\nu} a^\nu\right)
+
\text{(kinematical squares)}.
\label{eq:temp-constraint-expanded}
\end{equation}

This constraint is purely structural:  
it contains *no* independent function beyond geometric and congruence data.

% ------------------------------------------------------------
\subsection{Bondi limit and the IR constraint}

At future null infinity, the temporal connection reduces to its angular  
component $\Atemp_A$ via
\begin{equation}
\A_A(u,r,x^A) = \frac1r\,\Atemp_A(u,x^A) + O(r^{-2}).
\end{equation}

The leading-order constraint becomes
\begin{equation}
D^A \Atemp_A
=
D^2\Re(\sigma^0)
+
(\text{matter flux}),
\label{eq:temp-constraint-bondi}
\end{equation}
where $D^2$ is the Laplacian on the sphere.

Thus the IR temporal constraint is equivalent to the  
divergence of the Bondi shear’s real (E-mode) part.

No freedom remains beyond $\sigma^0$:  
\[
\Atemp_A \quad\text{contains no new DOF}.
\]

% ------------------------------------------------------------
\subsection{Interpretation}

\begin{quote}
Law I is the temporal analogue of GR's Hamiltonian constraint:  
it imposes an instantaneous relation tying the observable temporal  
connection to curvature and congruence geometry.

At $\mathscr I^+$, it reproduces the Bondi–Sachs shear constraint,  
confirming that the temporal soft mode is fully determined by the  
metric’s radiative data.
\end{quote}

% ================================================================



% ================================================================
\section{Law II: Temporal Raychaudhuri Evolution}
\label{sec:Law2_raychaudhuri}

\begin{quote}
\textbf{Law II (Temporal Raychaudhuri Evolution).}
Define the temporal expansion scalar
\begin{equation}
\Theta_{\mathrm T} := \nabla_\mu \A^\mu.
\end{equation}
Along the observer congruence $U^\mu$, $\Theta_{\mathrm T}$ obeys a
Raychaudhuri-type evolution equation:
\begin{align}
U^\rho \nabla_\rho \Theta_{\mathrm T}
&=
-\tfrac12 \Theta_{\mathrm T}^2
-\sigma^{\mathrm T}_{\mu\nu} \sigma_{\mathrm T}^{\mu\nu}
+\omega^{\mathrm T}_{\mu\nu} \omega_{\mathrm T}^{\mu\nu}
-R_{\mu\nu}\A^\mu U^\nu
-\F_{\mu\nu}\F^{\mu\nu}
+ \mathcal{S}_{\rm kin},
\label{eq:temporal-raychaudhuri}
\end{align}
where $\sigma^{\mathrm T}_{\mu\nu}$ and $\omega^{\mathrm T}_{\mu\nu}$
are the shear and vorticity of the temporal connection,  
$\F_{\mu\nu} = 2\nabla_{[\mu}\A_{\nu]}$ is the temporal curvature,  
and $\mathcal S_{\rm kin}$ contains lower-derivative kinematical terms.
\end{quote}

% ------------------------------------------------------------
\subsection{Kinematical decomposition}

Project the derivative of $\A_\mu$ onto the spatial hypersurface:
\begin{equation}
D_\mu \A_\nu
:= h_\mu{}^\rho h_\nu{}^\sigma \nabla_\rho \A_\sigma.
\end{equation}

Decompose into irreducible parts:
\begin{equation}
D_\mu\A_\nu
=
\frac12 h_{\mu\nu}\Theta_{\mathrm T}
+ \sigma^{\mathrm T}_{\mu\nu}
+ \omega^{\mathrm T}_{\mu\nu},
\label{eq:temp-kinematic-decomp}
\end{equation}
where  
$\sigma^{\mathrm T}_{\mu\nu}$ is symmetric trace-free,  
$\omega^{\mathrm T}_{\mu\nu}$ antisymmetric.

The temporal expansion $\Theta_{\mathrm T}$ measures the longitudinal rate  
of change of temporal transport.

% ------------------------------------------------------------
\subsection{Derivation outline}

Starting from  
\[
\Theta_{\mathrm T} = \nabla_\mu \A^\mu,
\]
differentiate along $U^\mu$:
\[
U^\rho \nabla_\rho \Theta_{\mathrm T}
= \nabla_\mu (U^\rho \nabla_\rho \A^\mu)
- (\nabla_\mu U^\rho)(\nabla_\rho \A^\mu).
\]

Use:

- the Ricci identity   
  \[
  \nabla_{[\mu}\nabla_{\rho]}\A_\nu
  = \tfrac12 R_{\mu\rho\nu}{}^\sigma \A_\sigma,
  \]

- the $1{+}3$ decomposition of $\nabla_\mu U_\nu$ into  
  $(\expn,\shear_{\mu\nu},\vort_{\mu\nu},a_\mu)$,

- the spatial decomposition \eqref{eq:temp-kinematic-decomp},

and group all terms into:  

- quadratic temporal-kinematic terms  
- curvature focusing terms  
- the $\F_{\mu\nu}\F^{\mu\nu}$ contribution  
- lower-derivative source terms $\mathcal S_{\rm kin}$.

This yields \eqref{eq:temporal-raychaudhuri}.

% ------------------------------------------------------------
\subsection{Physical meaning}

\begin{itemize}
\item $\Theta_{\mathrm T}^2$ and $\sigma_{\mathrm T}^2$  
      \(\rightarrow\) temporal focusing (analogous to geodesic convergence).

\item $\omega_{\mathrm T}^2$  
      \(\rightarrow\) temporal defocusing (rotational spreading in clock transport).

\item $R_{\mu\nu}\A^\mu U^\nu$  
      \(\rightarrow\) focusing from Ricci curvature and matter.

\item $\F_{\mu\nu}\F^{\mu\nu}$  
      \(\rightarrow\) focusing/defocusing from temporal curvature itself.

\item $\mathcal S_{\rm kin}$  
      \(\rightarrow\) contributions from the observer congruence.
\end{itemize}

% ------------------------------------------------------------
\subsection{Interpretation}

\begin{quote}
Law II shows that the temporal connection behaves like a  
\emph{derived congruence} with its own focusing equation.

Temporal expansion evolves according to curvature and shear, analogously  
to how geodesic congruences evolve under the usual Raychaudhuri equation.  

This law underpins temporal stability (Law 53) and cosmological Genesis  
(Law 33), and constrains interior reconstruction (Laws 15–17).
\end{quote}

% ================================================================


% ================================================================
\section{Law III: UV--IR Temporal Flux Law}
\label{sec:Law3_flux}

\begin{quote}
\textbf{Law III (UV--IR Temporal Flux Law).}
For any asymptotically-flat spacetime, the temporal soft charge
\[
Q_\Lambda(u) \;=\; \frac{1}{8\pi} \int_{S^2}
\Lambda(\Omega)\,D^A\mathcal{A}_A(u,\Omega)\,d\Omega
\]
obeys the universal flux-balance equation:
\begin{equation}
Q_\Lambda(u_2)-Q_\Lambda(u_1)
=
\frac{1}{8\pi}
\int_{u_1}^{u_2} \! du
\int_{S^2}
\Lambda(\Omega)\,B_A(u,\Omega)\,d\Omega
\;+\; \mathcal{F}^{\rm matter}_\Lambda,
\label{eq:temp-flux}
\end{equation}
where $B_A=\partial_u\mathcal{A}_A$ is the temporal news.
\end{quote}

The matter contribution is
\begin{equation}
\mathcal{F}^{\rm matter}_\Lambda
=
\frac{1}{8\pi}
\int_{\mathcal N(u_1,u_2)}
\Lambda(\Omega)\,
\big(8\pi\,T_{\mu\nu}U^\mu\A^\nu\big)\,dV,
\label{eq:matter-flux}
\end{equation}
integrated over the spacetime slab $\mathcal N$ bounded by the retarded
cuts $u_1,u_2$ of $\mathscr I^+$.

% ------------------------------------------------------------
\subsection{1. Derivation from the Bianchi identity}

The temporal curvature
\[
\F_{\mu\nu}=2\nabla_{[\mu}\A_{\nu]},
\]
satisfies
\[
\nabla_{[\lambda}\F_{\mu\nu]}=0.
\]

Project along the null generators of $\mathscr I^+$,
use the Bondi expansion for the shear, and integrate over a retarded 
time interval.  
One obtains the conservation law:
\[
\partial_u Q_\Lambda(u)
=
\frac{1}{8\pi}
\int_{S^2}\Lambda B_A\,d\Omega
+ \int_{S^2}\Lambda\,T^{\rm eff}_{uA}\,d\Omega,
\]
which integrates to \eqref{eq:temp-flux}.

The matter term comes from the effective flux  
\[
T^{\rm eff}_{uA} = T_{\mu\nu}k^\mu h^\nu{}_A,
\]
pulled to $\mathscr I^+$.

% ------------------------------------------------------------
\subsection{2. Interpretation}

\begin{itemize}
\item The \emph{temporal news} $B_A$ plays the role for the temporal
sector analogous to Bondi news for gravitational radiation.

\item The soft charge variation depends only on IR data
$(\Atemp,B)$ — no UV regulators enter.

\item Matter flux supplements gravitational temporal flux in the same
way EM current supplements Maxwell charge flux.
\end{itemize}

\begin{quote}
\emph{
Law III is the bridge between UV physics and IR observables:
any UV completion of quantum gravity must reproduce this flux-balance
equation in the infrared limit (see Law 30).
}
\end{quote}

% ------------------------------------------------------------
\subsection{3. Relation to memory}

Integrating \eqref{eq:temp-flux} with $\Lambda=1$ on a spherical patch:
\[
Q(u_2)-Q(u_1)
=
\frac{1}{8\pi}\int_{u_1}^{u_2}\! du\!\int_{S^2}B_A\,d\Omega
+\mathcal{F}^{\rm matter}.
\]

The term involving $B_A$ is precisely the temporal memory integral:
\[
\Delta \Atemp_A
=
-\int B_A\,du.
\]

Thus:
\[
\Delta Q_\Lambda
\propto
\int \Lambda\,\Delta \Atemp_A.
\]

This underlies the temporal memory law used in black-hole (Laws 15–22)
and cosmological (Laws 33–40) contexts.

% ------------------------------------------------------------
\subsection{4. Summary}

\begin{quote}
\emph{
Law III states that temporal charge is conserved up to flux of temporal
news and matter.  
It is the IR conservation law underlying temporal soft hair,  
temporal memory, and the entire UV–IR matching structure of the theory.
}
\end{quote}

% ================================================================


% ================================================================
\section{Law IV: Temporal Soft Symmetry}
\label{sec:Law4_soft}

\begin{quote}
\textbf{Law IV (Temporal Soft Symmetry).}
The infrared temporal potential $\Atemp_A(u,\Omega)$, defined as the
spin--1 projection of the Bondi shear,
\[
\Atemp_A(u,\Omega)
=
C_{AB}(u,\Omega)\,\bar m^B
+
\bar C_{AB}(u,\Omega)\,m^B,
\]
admits an infinite-dimensional family of angle-dependent symmetry
transformations:
\begin{equation}
\delta_\Lambda \Atemp_A = D_A \Lambda(\Omega),
\qquad
\ell(\Lambda)\ge 2,
\label{eq:temp-soft-sym}
\end{equation}
where $D_A$ is the covariant derivative on the sphere.
This transformation is a pure-\,$u$\,independent shift of the temporal IR
potential and leaves the physical curvature and news invariant:
\[
\delta_\Lambda B_A = 0,
\qquad
\delta_\Lambda \F_{\mu\nu}^{(0)}=0.
\]
\end{quote}

% ------------------------------------------------------------
\subsection{1. Origin of the symmetry}

Because the temporal field is \emph{derived} from the Bondi shear,  
the transformation \eqref{eq:temp-soft-sym} corresponds to adding an
$\ell\ge 2$ pure-gauge E-mode deformation of the shear:
\[
C_{AB}\;\rightarrow\; C_{AB}+D_A D_B \Lambda - \frac12 \gamma_{AB} D^2\Lambda.
\]

This shift:
\begin{itemize}
\item leaves the Bondi news $N_{AB}=\partial_u C_{AB}$ invariant,
\item changes only the *zero-frequency* or *memory* component,
\item produces a **nontrivial action** on IR observables
even though it is gauge-like at finite $u$.
\end{itemize}

Thus, it is a genuine **soft symmetry** in the sense of Strominger’s
analysis of large-gauge transformations.

% ------------------------------------------------------------
\subsection{2. Temporal soft charge}

Corresponding to \eqref{eq:temp-soft-sym} is the conserved charge:
\begin{equation}
Q_\Lambda(u)
=
\frac{1}{8\pi}
\int_{S^2}
\Lambda(\Omega)\,D^A \Atemp_A(u,\Omega)\,d\Omega,
\label{eq:temp-soft-charge}
\end{equation}
defined purely from gravitational IR data.

The charge acts through:
\[
\{Q_\Lambda, \Atemp_A\}
=
D_A\Lambda,
\]
which means **Law IV + Law VI** will form the canonical soft algebra.

% ------------------------------------------------------------
\subsection{3. Gauge-invariant quantities}

Because:
\[
\delta_\Lambda B_A = \delta_\Lambda \partial_u\Atemp_A = 0,
\]
the temporal **news**, which encodes radiative content,  
is invariant.  
Thus temporal symmetry acts exclusively on IR vacuum configurations:
\[
\Atemp_A \sim \Atemp_A + D_A\Lambda,
\]
while leaving local curvature unchanged.

This matches precisely the behavior of:
- BMS supertranslations (spin–0 sector),
- soft photons in QED (spin–1),
- soft gravitons (spin–2).

But here it is the **spin–1 part of the Bondi shear**, not a new field.

% ------------------------------------------------------------
\subsection{4. Physical interpretation}

\begin{quote}
\emph{
Temporal soft symmetry is a redundancy among IR vacua of gravity,
generated by the part of the Bondi shear that contributes to temporal
holonomy but carries no radiative energy.  
}
\end{quote}

It corresponds to:
- shifting the zero-frequency temporal profile,
- changing the accumulated temporal memory,
- modifying the phase of temporal holonomy around loops.

The symmetry does \textit{not} introduce a new gauge field;  
it reorganizes the IR gravitational data into a canonical form.

% ------------------------------------------------------------
\subsection{5. Structural role in the Temporal Sector}

Law IV is one of the three pillars of the IR temporal structure:

\[
\textbf{(i) Soft Symmetry (Law IV)}
\;\oplus\;
\textbf{(ii) Temporal Algebra (Law VI)}
\;\oplus\;
\textbf{(iii) Flux Balance (Law III)}.
\]



Together these:
- define the IR phase space,
- enforce Bianchi-identity-based conservation,
- constrain UV completions (Law 30).

% ------------------------------------------------------------
\subsection{6. Summary}

\begin{quote}
\emph{
Temporal soft symmetry is the fundamental IR redundancy of the derived
temporal sector: a pure E-mode shear shift that acts nontrivially on
memory and soft charges, but introduces no new dynamics.
}
\end{quote}

% ================================================================


% ================================================================
\section{Law V: Temporal Algebra and Noether Structure}
\label{sec:Law5_algebra}

\begin{quote}
\textbf{Law V (Temporal Algebra and Noether Structure).}
The infrared temporal fields $(\Atemp_A,B_A)$ derived from the Bondi
shear form a closed gravitational phase-space algebra determined entirely
by the Einstein–Hilbert symplectic form.  
The Noether (Wald–Zoupas) charge associated with the temporal soft
symmetry generates the correct canonical transformation on IR variables.
\end{quote}

% ------------------------------------------------------------
\subsection{1. IR temporal phase space from shear/news}

Define the temporal IR variables as:
\[
A_A(u,\Omega)=
C_{AB}(u,\Omega)\bar m^B+\bar C_{AB}(u,\Omega)m^B,
\qquad
B_A=\partial_u A_A.
\]

From the Bondi shear symplectic form,
\[
\Omega_{\rm grav}
=
\frac{1}{16\pi}
\int_{S^2}\!
\delta C^{AB}
\wedge
\delta N_{AB}\,d\Omega,
\]
we obtain the induced temporal symplectic form:
\[
\Omega_{\rm temp}
=
\frac{1}{16\pi}
\int_{S^2}
\delta A_A
\wedge
\delta B^A\,d\Omega.
\]

Thus the temporal fields inherit their entire algebra from gravity —  
*no new terms, no independent action*.

% ------------------------------------------------------------
\subsection{2. Canonical brackets}

From $\Omega_{\rm temp}$ we read off the Poisson brackets:
\begin{equation}
\{A_A(\Omega),B_B(\Omega')\}
=
16\pi\,
\gamma_{AB}\,
\delta^{(2)}(\Omega,\Omega'),
\label{eq:AB-bracket}
\end{equation}
and all other equal-time brackets vanish.

This is the *full* classical algebra of the temporal IR sector.

% ------------------------------------------------------------
\subsection{3. Noether charge generating the soft symmetry}

The soft charge from Law IV is:
\[
Q_\Lambda
=
\frac{1}{8\pi}
\int_{S^2}
\Lambda\,D^A A_A\,d\Omega.
\]

Using bracket \eqref{eq:AB-bracket}:
\[
\{Q_\Lambda,A_A\}=D_A\Lambda,
\qquad
\{Q_\Lambda,B_A\}=0,
\]
exactly reproducing the temporal soft symmetry transformation:
\[
\delta_\Lambda A_A=D_A\Lambda.
\]

Thus the soft symmetry is \emph{canonical}, not imposed by hand.

% ------------------------------------------------------------
\subsection{4. Closure of the algebra}

Because:
\[
\{Q_\Lambda,Q_{\Lambda'}\}=0,
\]
the temporal soft sector is an infinite-dimensional Abelian algebra
(same structure as QED electric large gauge symmetries or
supertranslations projected to spin–1 channel).

There is no central extension:
- because the temporal field is shear-derived,  
- and because $A_A$ has no independent dynamical phase.

This is consistent with Axiom 0.

% ------------------------------------------------------------
\subsection{5. Noether–Wald–Zoupas structure}

The Wald–Zoupas charge variation for the gravitational shear is:
\[
\delta Q_\Lambda^{\rm WZ}
=
\frac{1}{16\pi}
\int_{S^2}
\Lambda\,D^A \delta A_A\,d\Omega.
\]

Integrating over phase space reproduces the soft charge \eqref{eq:AB-bracket}.

Thus:
- the *Noether charge*,  
- the *canonical generator*,  
- and the *soft memory charge*  

are all the same object viewed through different lenses.

This triple identification is the hallmark of infrared gravitational structure.

% ------------------------------------------------------------
\subsection{6. Interpretation: algebra = kinematics of soft time}

\begin{quote}
\emph{
The temporal algebra is simply the gravitational shear/news algebra written
in a spin–1 basis.  
It quantifies the IR kinematics of gravitational time transport, not the
dynamics of a new field.
}
\end{quote}

Thus the temporal sector:
- is canonical,
- closed,
- derived,
- gauge-constrained,
- and entirely contained within GR's radiative phase space.

% ------------------------------------------------------------
\subsection{7. Summary}

\begin{quote}
\emph{
Law V establishes the canonical phase-space structure of the temporal
sector:  
the brackets, the Noether charge, and the soft symmetry form a single
geometric package inherited from gravitational radiation.
}
\end{quote}

% ================================================================

% ================================================================
\section{Law VI: IR Temporal Phase Space Law}
\label{sec:Law6_phase_space}

\begin{quote}
\textbf{Law VI (IR Temporal Phase Space).}
The infrared temporal variables $(A_A,B_A)$ defined as the spin–1
projection of the Bondi shear and news form a complete canonical
subsystem of the gravitational radiative phase space.  
Their symplectic form and Hamiltonian flows are inherited entirely from
the Einstein–Hilbert symplectic structure, with no additional degrees of
freedom introduced.
\end{quote}

% ------------------------------------------------------------
\subsection{1. IR variables from Bondi shear}

The Bondi shear and news are decomposed into spin–$\pm2$ modes:
\[
C_{AB} = \sigma^0\,\bar m_A \bar m_B + \bar\sigma^0\,m_A m_B,
\qquad
N_{AB} = \partial_u C_{AB}.
\]

Define the temporal spin–1 projection:
\[
A_A = -\left(\sigma^0\,\bar m_A + \bar\sigma^0\,m_A\right),
\qquad
B_A = \partial_u A_A.
\]

Because these are linear combinations of $C_{AB}$ and $N_{AB}$,
their phase-space structure is fully inherited.

% ------------------------------------------------------------
\subsection{2. Symplectic form}

The Einstein–Hilbert radiative symplectic form is:
\[
\Omega_{\rm grav}
=
\frac{1}{16\pi}\!
\int_{S^2}
\delta C^{AB}\wedge\delta N_{AB}\,d\Omega.
\]

Projecting onto the spin–1 temporal sector yields:
\[
\Omega_{\rm temp}
=
\frac{1}{16\pi}
\int_{S^2}
\gamma^{AB}\,
\delta A_A\wedge\delta B_B\,d\Omega,
\]
where $\gamma_{AB}$ is the unit-sphere metric.

This is the \emph{complete} symplectic structure of the temporal IR mode.

% ------------------------------------------------------------
\subsection{3. Canonical Poisson brackets}

From $\Omega_{\rm temp}$ one obtains:
\begin{equation}
\{A_A(\Omega),B_B(\Omega')\}
=
16\pi\,\gamma_{AB}\,\delta^{(2)}(\Omega,\Omega'),
\label{eq:ABbrackets}
\end{equation}
and all other equal-time brackets vanish.

Thus $(A_A,B_A)$ are canonical conjugates.

% ------------------------------------------------------------
\subsection{4. Hamiltonian flows}

For any functional $H[A,B]$,
the temporal evolution generated by $H$ is:
\[
\delta_H A_A = \{A_A,H\} 
= 16\pi\,\gamma_{AB}\frac{\delta H}{\delta B_B},
\]
\[
\delta_H B_A = \{B_A,H\}
= -16\pi\,\gamma_{AB}\frac{\delta H}{\delta A_B}.
\]

In particular:

- the Bondi time evolution $u\mapsto u+\delta u$  
- the modular evolution (Law VII)  

are both Hamiltonian flows on this IR phase space.

There is *no independent Hamiltonian for the temporal sector* —  
all flows arise from gravitational dynamics.

% ------------------------------------------------------------
\subsection{5. Gauge and kernel structure}

Because $A_A = D_A \Phi_{\rm temp}$ for some scalar potential (up to
$\ell=1$ modes), the symplectic form is degenerate along:

- $\ell=0,1$ kernel (supertranslation-like zero modes),
- pure-gauge shifts $A_A \to A_A + D_A\Lambda$.

This is the same degeneracy appearing in the full gravitational
radiative phase space projected onto spin–1.

Thus the temporal IR sector is a **quotient phase space**, not a new one.
% ------------------------------------------------------------
\subsection{6. Relationship to Law V (soft algebra)}
From the bracket \eqref{eq:ABbrackets},
the soft charge 
\[
Q_\Lambda = \frac{1}{8\pi}\!\int\Lambda D^A A_A\,d\Omega
\]
generates:
\[
\{Q_\Lambda,A_A\}=D_A\Lambda,
\qquad
\{Q_\Lambda,B_A\}=0,
\]
verifying:
- closure of the soft algebra,
- canonical nature of the symmetry,
- consistency with the Bondi shear projection.

Thus Law V and Law VI together establish:
\[
\text{(soft symmetry)} 
\quad+\quad
\text{(canonical phase space)}
\quad+\quad
\text{(IR boundary field)}
\]
as a single coherent GR structure.

% ------------------------------------------------------------
\subsection{7. Interpretation}

\begin{quote}
\emph{
Law VI asserts that the temporal IR variables form a complete canonical
sector of gravitational radiation.  
Their phase space, symplectic structure, and Hamiltonian evolution are
entirely inherited from the $(C_{AB},N_{AB})$ sector of GR.
No extra DOFs exist; no new action is required.
}
\end{quote}

This closes the derivation of the classical temporal kinematics.

% ================================================================


% ================================================================
\section{Law VII: Temporal Modular Hamiltonian Law}
\label{sec:Law7_modular}

\begin{quote}
\textbf{Law VII (Temporal Modular Hamiltonian).}
The infrared temporal variables $(A_A,B_A)$ admit a canonical quadratic
functional --- the \emph{Temporal Modular Hamiltonian} --- whose
Hamiltonian flow reorganises temporal information and defines a natural
monotonic functional of time.  
This modular structure is inherited from the gravitational radiative
sector and introduces no new degrees of freedom.
\end{quote}

% ------------------------------------------------------------
\subsection{1. Definition}

Given the canonical phase space of Law~VI,
\[
\{A_A(\Omega), B_B(\Omega')\}
=
16\pi\,\gamma_{AB}\,\delta^{(2)}(\Omega,\Omega'),
\]
define the Temporal Modular Hamiltonian:
\begin{equation}
K_{\rm T}[A,B]
=
\frac{1}{16\pi}
\int_{S^2}
\left(
B^A B_A 
+ 
\lambda\, D^A A^B D_A A_B
\right) d\Omega ,
\label{eq:KTdef}
\end{equation}
where $\lambda>0$ is a dimensionless numerical constant fixed by
normalisation of the temporal sector.

It is fully determined by the projection $A_A=-C_{AB}q^B$ and
$B_A=\partial_u A_A=-N_{AB}q^B$ with $q^B$ the spin–1 dyad.

% ------------------------------------------------------------
\subsection{2. Modular flow equations}

The Hamiltonian flow generated by $K_{\rm T}$ is:
\[
\partial_\tau A_A
= 
\{A_A, K_{\rm T}\}
=
B_A + \lambda\,\Delta_{S^2} A_A,
\]
\[
\partial_\tau B_A
= 
\{B_A, K_{\rm T}\}
=
\lambda\,\Delta_{S^2} B_A,
\]
where $\Delta_{S^2}$ is the Laplacian on the sphere.

This flow does \emph{not} describe physical Bondi evolution in $u$;
it is a canonical rearrangement of IR temporal data.

% ------------------------------------------------------------
\subsection{3. Monotonicity of modular functionals}

Define the temporal soft entropy:
\begin{equation}
S_{\rm temp}[A_A]
=
\frac{1}{16\pi}
\int_{S^2} 
A^A (-\Delta_{S^2}) A_A\,d\Omega.
\label{eq:tempEntropy}
\end{equation}

Under modular flow:
\[
\frac{d}{d\tau} S_{\rm temp} \ge 0.
\]

Thus the modular parameter $\tau$ provides a natural ``arrow of time’’
on the IR temporal phase space.

% ------------------------------------------------------------
\subsection{4. GR origin (no new DOF)}

The modular Hamiltonian arises as a quadratic functional of the Bondi
shear:
\[
A_A \sim \sigma^0,\qquad 
B_A \sim \partial_u \sigma^0 = N.
\]

Therefore:
- $K_{\rm T}$ is a \emph{boundary functional of the GR radiative data};
- It does not introduce new fields, equations, or dynamics;
- All modular evolution is a canonical transformation of existing GR
degrees of freedom.

% ------------------------------------------------------------
\subsection{5. Interpretation}

\begin{quote}
\emph{
Law VII elevates the temporal IR variables into a canonical modular
structure analogous to modular Hamiltonians in algebraic QFT—but here it
is entirely classical and derived from the Bondi shear.  
It defines a natural entropy functional and a monotonic temporal
parameter, creating an intrinsic ``IR arrow of time’’ inside GR.
}
\end{quote}

% ================================================================


% ================================================================
\section{Law VIII: Temporal Relative Entropy and the Arrow of Time}
\label{sec:Law8_entropy}

\begin{quote}
\textbf{Law VIII (Temporal Relative Entropy Law).}
The temporal IR sector possesses a natural relative entropy functional,
constructed from the canonical variables $(A_A,B_A)$ and the Temporal
Modular Hamiltonian $K_{\rm T}$.  
This relative entropy is monotonic under modular flow, establishing an
intrinsic ``arrow of time’’ entirely within the infrared gravitational
sector.
\end{quote}

% ---------------------------------------------------------------
\subsection{1. Temporal phase-space distribution}

Let $\Gamma_{\rm temp}$ denote the temporal IR phase space with
coordinates $(A_A,B_A)$ and canonical symplectic measure $d\Gamma$ from
Law~VI.  
A coarse-grained distribution on this space is a functional
$\rho[A,B]$ satisfying:
\[
\rho\ge0,\qquad 
\int_{\Gamma_{\rm temp}} \rho\, d\Gamma = 1.
\]

This distribution represents coarse-grained information about the Bondi
shear and news, with no new physical degrees of freedom introduced.

% ---------------------------------------------------------------
\subsection{2. Reference (modular) state}

Define the ``temporal vacuum’’ distribution by:
\begin{equation}
\rho_0[A,B]
\propto \exp\!\left(-K_{\rm T}[A,B]\right),
\label{eq:rho0def}
\end{equation}
where $K_{\rm T}$ is the Temporal Modular Hamiltonian of Law~VII.

This is the classical analogue of a Gibbs-like state generated by a
quadratic functional of the Bondi shear.

% ---------------------------------------------------------------
\subsection{3. Temporal relative entropy}

Define the temporal relative entropy of $\rho$ with respect to $\rho_0$:
\begin{equation}
S_{\rm rel}(\rho \| \rho_0)
=
\int_{\Gamma_{\rm temp}}
\rho[A,B]\,
\ln\!\left(\frac{\rho[A,B]}{\rho_0[A,B]}\right)
d\Gamma.
\label{eq:Srel}
\end{equation}

Properties:
\[
S_{\rm rel}(\rho\|\rho_0)\ge 0,
\qquad 
S_{\rm rel}=0 \iff \rho=\rho_0.
\]

Thus $\rho_0$ is the ``minimal information’’ temporal configuration.

% ---------------------------------------------------------------
\subsection{4. Monotonicity under modular flow}

Let $\partial_\tau$ denote the modular flow generated by $K_{\rm T}$:
\[
\partial_\tau A_A 
= \{A_A, K_{\rm T}\},\qquad
\partial_\tau B_A 
= \{B_A, K_{\rm T}\}.
\]

Then:
\begin{equation}
\frac{d}{d\tau} 
S_{\rm rel}(\rho(\tau)\|\rho_0)
\le 0.
\label{eq:SrelMonotonic}
\end{equation}

Thus modular flow drives the temporal IR sector toward the reference
distribution $\rho_0$.

This is purely a statement about the phase-space geometry of Bondi shear
projections; no dynamical propagation is implied.

% ---------------------------------------------------------------
\subsection{5. Emergent arrow of time}

Since $S_{\rm rel}$ is non-increasing while the Temporal Soft Entropy
$S_{\rm temp}$ of Law~VII is non-decreasing:
\[
\frac{d}{d\tau} S_{\rm rel} \le 0,
\qquad 
\frac{d}{d\tau} S_{\rm temp} \ge 0,
\]
the modular parameter $\tau$ defines an intrinsic, canonical ``IR arrow
of time’’ inside asymptotically-flat gravity.

This arrow arises not from matter thermodynamics but from the structure
of gravitational radiation and its spin–1 temporal projection.

% ---------------------------------------------------------------
\subsection{6. Interpretation}

\begin{quote}
\emph{
Law VIII shows that the temporal sector of general relativity contains
its own entropic arrow of time, based purely on the infrared radiative
shear.  
This provides a classical gravitational counterpart to the monotonicity
of relative entropy in quantum field theory.  
No new fields are introduced; the law reorganises information already
present in GR.
}
\end{quote}

% ================================================================


% ===================================================================
\section{Law IX: Temporal Soft Holographic Dictionary}
\label{sec:Law9_holography}

\begin{quote}
\textbf{Law IX (Temporal Soft Holographic Dictionary).}
The infrared temporal fields
\(
\Atemp_A(u,\Omega),\;
B_A(u,\Omega)=\partial_u\Atemp_A
\)
constitute a complete boundary encoding of the \emph{causally accessible}
information contained in the temporal curvature
\(
\F_{\mu\nu}=2\nabla_{[\mu}\A_{\nu]}
\)
in the bulk.
There exists a pair of retarded kernels $(R,R')$ such that, for any
interior point $x$ lying in the causal past of $\mathscr I^+$,
\[
\boxed{
\F_{\mu\nu}(x)
=
\int_{\mathscr I^+}
\!\!\left[
R_{\mu\nu}{}^{A}(x|u,\Omega)\,
\Atemp_A(u,\Omega)
+
R'_{\mu\nu}{}^{A}(x|u,\Omega)\,
B_A(u,\Omega)
\right]du\,d\Omega.
}
\]
\end{quote}

% -------------------------------------------------------------------
\subsection{1. Bulk-to-boundary projection (Bondi limit)}

The temporal connection admits the asymptotic expansion
\[
\A_A(u,r,\Omega)
=
r^{-1}\Atemp_A(u,\Omega)
+O(r^{-2}),
\]
where
\(
\Atemp_A
\)
is a spin--1 projection of the Bondi shear,
$
\Atemp_A
= -\sigma^0 \bar m_A - \bar\sigma^0 m_A.
$

Thus the temporal IR variables contain no new degrees of freedom; they
repackage gravitational radiation.

The bulk curvature projects as:
\[
\Atemp_A(u,\Omega)
=
\int d^4x'\;
K_A{}^{\mu\nu}(u,\Omega|x')\,
\F_{\mu\nu}(x'),
\]
with $K_A{}^{\mu\nu}$ determined entirely by Bondi asymptotics.

% -------------------------------------------------------------------
\subsection{2. Boundary-to-bulk reconstruction (temporal HKLL analogue)}

For any bulk observable $\mathcal{O}(x)$ that is:
\begin{itemize}
\item gauge-invariant under diffeomorphisms preserving Bondi gauge, and
\item supported inside the causal domain of $\mathscr I^+$,
\end{itemize}
there exist kernels $(R,R')$ such that:
\begin{equation}
\mathcal{O}(x)
=
\int_{\mathscr I^+}
\Big[
R(x|u,\Omega)\,\Atemp_A(u,\Omega)
+
R'(x|u,\Omega)\,B_A(u,\Omega)
\Big]
du\,d\Omega.
\label{eq:bulkobsrecon}
\end{equation}

This is the temporal analogue of HKLL reconstruction in AdS, but here
arising entirely from the $r\to\infty$ structure of asymptotically flat
GR.

% -------------------------------------------------------------------
\subsection{3. Causal consistency: where reconstruction is allowed}

Reconstruction is valid precisely when $x$ lies in the domain
$J^{-}(\mathscr I^+)$ where:
\[
\Theta_{\rm T}>0,
\qquad 
\F_{\mu\nu}\F^{\mu\nu}<\infty,
\]
ensuring temporal transport is non-singular along rays reaching $x$.

Points behind caustics or strong-focusing regions cannot be reconstructed,
preserving causal and geometric consistency.

% -------------------------------------------------------------------
\subsection{4. Temporal soft two-point dictionary}

Correlators of bulk temporal curvature reduce to boundary correlators:
\[
\langle\F_{\mu\nu}(x)\,\F_{\rho\sigma}(y)\rangle
=
\int_{\mathscr I^+\times \mathscr I^+}
\!\!\!\!\!\!\!
R_{\mu\nu}{}^{A}(x|u,\Omega)\;
R_{\rho\sigma}{}^{B}(y|u',\Omega')
\;
\langle \Atemp_A(u,\Omega)\,B_B(u',\Omega')\rangle
\,du\,d\Omega\,du'\,d\Omega'.
\]

Since $\Atemp_A$ and $B_A$ are projections of shear/news,
\emph{all temporal bulk correlators are gravitational shear correlators
in disguise}.

% -------------------------------------------------------------------
\subsection{5. Interpretation}

\begin{quote}
\emph{
Law~IX establishes that the temporal sector provides a genuine soft
holographic channel in asymptotically flat gravity.  
No new fields are introduced; the dictionary emerges from the Bondi
expansion and the causal structure of $\A_\mu$ derived from vorticity.
The temporal soft data serve as boundary ``coordinates’’ for the part of
the bulk curvature that is operationally accessible.
}
\end{quote}

% ===================================================================


% ===================================================================
\section{Law X: Temporal Soft Noether--Bianchi Identity}
\label{sec:Law10_Bianchi}

\begin{quote}
\textbf{Law X (Temporal Soft Noether--Bianchi Identity).}
The temporal soft symmetry, its associated charge $Q_\Lambda$, and
the corresponding memory effect follow necessarily from:
\[
\nabla_{[\mu} R_{\nu\rho]\sigma}{}^{\tau} = 0
\quad\text{(Bianchi identity)}
\qquad\text{and}\qquad
G_{\mu\nu}=8\pi T_{\mu\nu}.
\]
In particular, the quantity
\[
J^\mu_{\rm T}(\Lambda) := \nabla_\nu\!\left( \F^{\mu\nu}\Lambda \right),
\qquad
\F_{\mu\nu}=2\nabla_{[\mu}\A_{\nu]},
\]
defines a \emph{conserved current}:
\[
\boxed{
\nabla_\mu J^\mu_{\rm T} = 0 ,
}
\]
whose flux to $\mathscr I^+$ reproduces temporal soft charge
conservation and memory.
\end{quote}

% -------------------------------------------------------------------
\subsection{1. The temporal current from the curvature identity}

Use the commutator
\[
\nabla_{[\mu}\nabla_{\nu]}\A_{\rho}
=
\frac12 R_{\mu\nu\rho}{}^{\sigma}\A_{\sigma},
\]
with $\A_\mu = (2\alpha)^{-1}\omega_{\nu\mu}U^\nu$.
Contracting appropriately and integrating by parts yields the identity:
\[
\nabla_\mu\left( \F^{\mu\nu}\Lambda \right)
=
\F^{\mu\nu} \nabla_\mu\Lambda
+
\Lambda \nabla_\mu\F^{\mu\nu}.
\]

Einstein’s equations imply
\(
\nabla_\mu\F^{\mu\nu}
\)
projects onto combinations of matter fluxes and shear derivatives.  
The Bianchi identity ensures all such contributions combine to a total
derivative.

Therefore:
\[
\nabla_\mu J^\mu_{\rm T}=0,
\]
independently of the congruence gauge or Bondi frame chosen.

% -------------------------------------------------------------------
\subsection{2. Integration to future null infinity}

Integrating the conservation law over a region $\mathcal{N}$ bounded by
cuts $\mathcal{C}_{u_1},\mathcal{C}_{u_2}\subset \mathscr I^+$:
\[
0
=
\int_{\mathcal{N}}\nabla_\mu J^\mu_{\rm T}
=
\int_{\mathcal{C}_{u_2}} \!J^\mu_{\rm T} d\Sigma_\mu
-
\int_{\mathcal{C}_{u_1}} \!J^\mu_{\rm T} d\Sigma_\mu.
\]

The leading term of $J^\mu_{\rm T}$ near $\mathscr I^+$ is:
\[
J^u_{\rm T}
=
\frac{1}{8\pi}\,
\Lambda\,\partial_u\!\left(
D^A\Atemp_A
\right)
+O(r^{-1}),
\]
so one obtains the **soft charge conservation law**:
\[
\boxed{
Q_\Lambda(u_2)-Q_\Lambda(u_1)
=
\frac{1}{8\pi}\int_{u_1}^{u_2}\!\!\!du
\int_{S^2}\Lambda\,\partial_u A\,d\Omega .
}
\]

Using $B_A=\partial_u\Atemp_A$ gives the **temporal memory formula**:
\[
\Delta A(\Omega)
=
-\int_{u_1}^{u_2} B(u,\Omega)\,du.
\]

% -------------------------------------------------------------------
\subsection{3. Interpretation and necessity}

\begin{quote}
\emph{
Law~X shows that temporal soft symmetry, soft charge conservation,
and temporal memory are not optional additions to GR.
They arise necessarily from the Bianchi identity and the 
vorticity-defined temporal connection.
Any consistent treatment of the Bondi asymptotic structure must contain
the temporal soft sector.
}
\end{quote}

\[
\boxed{
\text{No temporal soft symmetry} 
\;\Longrightarrow\;
\text{Bianchi identity violation}.
}
\]

This makes the temporal soft sector as fundamental as the BMS
supertranslation symmetry of the spin–2 sector.

% ===================================================================


% ===================================================================
\section{Law XI: Temporal Hamilton--Jacobi Principle}
\label{sec:Law11_HJ}

\begin{quote}
\textbf{Law XI (Temporal Hamilton--Jacobi Principle).}
There exists a boundary functional
\[
S_{\rm temp}[C_{AB},\Atemp_A]
\]
defined on $\mathscr I^+$ such that:

1. Its functional derivative with respect to the Bondi shear produces  
   the temporal soft charge density:
\[
\boxed{
\frac{\delta S_{\rm temp}}{\delta \sigma^0(u,\Omega)}
=
\frac{1}{8\pi}\,D^A \Atemp_A(u,\Omega)
}
\]

2. Its on–shell variation between two scattering vacua reproduces the  
   temporal memory:
\[
\boxed{
\delta S_{\rm temp}\big|_{\rm on\text{-}shell}
=
\Delta Q_\Lambda
}
\]

Thus, temporal memory is the Hamilton–Jacobi variation of a boundary 
action for the derived temporal connection.
\end{quote}

% -------------------------------------------------------------------
\subsection{1. Construction of the temporal Hamilton--Jacobi functional}

Because $\Atemp_A$ is a spin--1 projection of the Bondi shear 
\[
\Atemp_A = -\sigma^0 \bar m_A - \bar\sigma^0 m_A,
\]
the simplest scalar functional consistent with the symmetry and 
dimensional analysis is:
\begin{equation}
S_{\rm temp}
:=
\frac{1}{8\pi}
\int_{\mathscr I^+}
\Atemp_A D^A \sigma^0
\;du\,d\Omega .
\label{eq:Stemp-def}
\end{equation}

This is a pure boundary term and introduces *no new degrees of freedom*,  
consistent with Axiom~0.

% -------------------------------------------------------------------
\subsection{2. Functional derivative and soft charge}

Varying \eqref{eq:Stemp-def} with respect to $\sigma^0$ gives:
\[
\delta S_{\rm temp}
=
\frac{1}{8\pi}
\int_{\mathscr I^+}
\delta\sigma^0\, D^A\Atemp_A\,du\,d\Omega,
\]
so:
\[
\frac{\delta S_{\rm temp}}{\delta \sigma^0}
=
\frac{1}{8\pi} D^A\Atemp_A,
\]
which is precisely the \emph{temporal soft charge density}.

Thus:
\[
Q_\Lambda
=
\int_{S^2}\Lambda \frac{\delta S_{\rm temp}}{\delta \sigma^0} d\Omega .
\]

% -------------------------------------------------------------------
\subsection{3. On--shell variation and memory}

On shell, the Bondi news obeys:
\[
\partial_u \sigma^0 = N,
\qquad
\partial_u \Atemp_A = B_A.
\]

For scattering between early and late vacua, 
the on--shell variation becomes:
\[
\delta S_{\rm temp}\big|_{\rm on\text{-}shell}
=
\frac{1}{8\pi}
\int_{S^2}
\Lambda\,\Delta\!\left(D^A\Atemp_A\right)
d\Omega ,
\]
which equals the temporal memory effect:
\[
\Delta A(\Omega)
=
- \int B(u,\Omega)\,du .
\]

Therefore:
\[
\boxed{
\delta S_{\rm temp}\big|_{\rm on\text{-}shell}
=
Q_\Lambda(u_f) - Q_\Lambda(u_i)
}
\]

% -------------------------------------------------------------------
\subsection{4. Interpretation}

\begin{quote}
\emph{
Law~XI reveals that temporal memory, flux, and soft charge conservation
can all be derived from a single boundary functional.  
The temporal sector therefore possesses a variational origin analogous to
the Hamilton--Jacobi treatment of classical mechanics and holographic
renormalisation.
}
\end{quote}

This ties together Laws I–X:
- Law I: constraint  
- Law II: evolution  
- Law III: flux  
- Law IV–V: symmetry + algebra  
- Law VI–VIII: canonical + entropic structure  
- Law IX: holography  
- Law X: conservation (Bianchi)  
- **Law XI: variational principle**

Together they complete the classical temporal sector.

% ===================================================================


% ================================================================
\section{Law XII: The Temporal Causal Wedge}
\label{sec:Law12_TCW}

\begin{quote}
\textbf{Law XII (Temporal Causal Wedge).}
For any asymptotically-flat black-hole spacetime, the temporal data on
$\mathscr I^+$ --- equivalently the spin--1 projection of the Bondi shear
$\Atemp_A$ and its news $B_A$ --- determines a \emph{boundary-defined, causally accessible interior region}:
\[
\boxed{
\mathcal{W}_{\rm TC}(u,\Omega)
=
J^{-}\!\Big(\gamma_{\rm temp}(u,\Omega)\Big)
\cap
\mathcal{M}_{\rm BH},
}
\]
called the \emph{Temporal Causal Wedge}.
All interior relational observables supported in $\mathcal{W}_{\rm TC}$
are reconstructible from the boundary temporal fields via retarded 
integral kernels, while no information outside $\mathcal{W}_{\rm TC}$ 
is accessible.

\end{quote}

% ----------------------------------------------------------------
\subsection{1. Temporal congruence and canonical interior curve}

Temporal parallel transport defines a preferred interior trajectory
through:
\begin{equation}
U^\mu_{\rm temp}\,\nabla_\mu U^\nu_{\rm temp}
=
\F^{\nu}{}_{\mu}\,U^\mu_{\rm temp},
\label{eq:temp-geodesic}
\end{equation}
where the temporal curvature $\F_{\mu\nu}=2\nabla_{[\mu}\A_{\nu]}$ is 
a derived, non-dynamical functional of the Bondi shear (Axiom~0).

The integral curve passing through $(u,\Omega)$ at $\mathscr I^+$ is
denoted $\gamma_{\rm temp}(u,\Omega)$.
It defines the ``temporal anchor'' of bulk reconstruction.

% ----------------------------------------------------------------
\subsection{2. Definition of the temporal causal wedge}

Given $\gamma_{\rm temp}(u,\Omega)$, the wedge is:
\begin{equation}
\mathcal{W}_{\rm TC}(u,\Omega)
=
J^{-}\!\left(\gamma_{\rm temp}(u,\Omega)\right)
\cap
\mathcal{M}_{\rm BH}.
\end{equation}

A bulk point $x$ lies in $\mathcal{W}_{\rm TC}$ iff:
\[
x \in J^{-}(\gamma_{\rm temp})
\quad\text{and}\quad
x \text{ is interior to the horizon}.
\]

This is the analogue of:
- the causal wedge in AdS/CFT,  
- the causal shadow region in null geometry,  
but defined using the \emph{temporal} congruence rather than null 
geodesics or extremal surfaces.

% ----------------------------------------------------------------
\subsection{3. Reconstruction inside the wedge}

Any relational bulk observable $\mathcal{O}(x)$ supported in
$\mathcal{W}_{\rm TC}$ can be reconstructed via:
\begin{equation}
\mathcal{O}(x)
=
\int_{\mathscr I^+}\!
\big[
R(x|u,\Omega)\,\Atemp_A
+
R'(x|u,\Omega)\,B_A
\big]
\,du\,d\Omega ,
\label{eq:recon-TCW}
\end{equation}
where $R$ and $R'$ are retarded kernels determined entirely by the
background spacetime.

Crucially:
- **no temporal field is introduced in the bulk**,  
- the reconstruction is purely kinematical,  
- and consistency follows from the Bondi constraint structure.

% ----------------------------------------------------------------
\subsection{4. Boundary accessibility conditions}

A point $x$ is reconstructible iff the temporal congruence does not
focus before reaching $x$:
\begin{equation}
\Theta_{\rm T} > 0,
\qquad
\F_{\mu\nu}\F^{\mu\nu} < \infty,
\label{eq:TCW-access}
\end{equation}
ensuring the temporal flow remains regular.

These conditions are inherited from the Raychaudhuri-type behaviour of
the derived temporal connection (Law~II).

% ----------------------------------------------------------------
\subsection{5. Interpretation}

\begin{quote}
\emph{
Law~XII states:  
\textbf{the interior region of a black hole that is reconstructible from the asymptotic temporal soft sector is precisely the Temporal Causal Wedge}.  
The wedge is determined not by new fields, but by the derived temporal
connection encoded in the Bondi shear.
}
\end{quote}

This provides:
- the starting point for temporal interior reconstruction (Law XIII),  
- constraints on black-hole information flow,  
- a boundary characterization of which interior points are observable.

% ================================================================


% ================================================================
\section{Inter-Law Structural Summary}
\label{sec:interlaw_summary}

The twelve Laws developed above constitute a complete structural
framework for the classical temporal sector of general relativity.
Because the temporal connection $\A_\mu$ is a derived functional of
$(g_{\mu\nu},U^\mu)$, the resulting Laws do not extend GR but rather
organise existing geometric information into a coherent subsystem.
This section summarises their interrelations.

% ---------------------------------------------------------------
\subsection{Constraint--Evolution Structure (Laws I–II)}

Law~I provides a constraint equation for $D^A\A_A$ on spacelike or
null hypersurfaces.  
Law~II complements this with a Raychaudhuri-type evolution equation for
the temporal expansion $\Theta_{\rm T}=\nabla_\mu\A^\mu$.
Together they define the instantaneous and dynamical relations among the
temporal curvature, vorticity, shear, and Ricci tensor.

This mirrors the familiar Hamiltonian--Raychaudhuri pair for $U^\mu$,
but applied to the derived temporal congruence.

% ---------------------------------------------------------------
\subsection{UV--IR and Soft Structure (Laws III–VI)}

Laws~III–VI concern the infrared temporal mode 
$\Atemp_A$ and its news $B_A$ at $\mathscr I^+$.

\begin{itemize}
\item \textbf{Law~III} gives the flux-balance relation for temporal
soft charges, relating temporal memory to the integrated temporal news
and matter flux.
\item \textbf{Law~IV} identifies the soft symmetry
$\delta_\Lambda A_A = D_A\Lambda$, inherited from E-mode variations of
Bondi shear.
\item \textbf{Law~V} shows that the corresponding Noether charge
$Q_\Lambda$ generates this symmetry and that the soft sector forms an
Abelian canonical algebra.
\item \textbf{Law~VI} elevates $(A_A,B_A)$ to a complete infrared phase
space obtained by projecting the Einstein--Hilbert symplectic form.
\end{itemize}

These four Laws demonstrate the entire IR temporal sector is contained in
the gravitational radiative phase space.  
It is not new physics but a refined basis for familiar soft-graviton
structure.

% ---------------------------------------------------------------
\subsection{Canonical and Entropic Structure (Laws VII–VIII)}

Laws~VII and VIII establish canonical and entropic functionals:
a Temporal Modular Hamiltonian $K_{\rm T}$ and a relative entropy
$S_{\rm rel}$.

\begin{itemize}
\item The modular Hamiltonian generates canonical flows within the
derived temporal phase space, organising IR information without
introducing dynamics.
\item The temporal entropy and relative entropy are monotonic under
modular flow, producing an intrinsic IR arrow of time.
\end{itemize}

This subsystem parallels algebraic quantum field theory, but remains
purely classical and derived from shear/news.

% ---------------------------------------------------------------
\subsection{Holography and Conservation (Laws IX–XI)}

Laws~IX--XI provide a soft holographic dictionary, a conserved temporal
current, and a Hamilton--Jacobi variational principle.

\begin{itemize}
\item \textbf{Law IX} reconstructs temporal curvature in the bulk from
boundary temporal data.
\item \textbf{Law X} shows that soft charge conservation and memory
follow directly from the Bianchi identity and Einstein’s equations.
\item \textbf{Law XI} constructs a boundary functional whose variation
gives the temporal soft charge density and whose on-shell difference
yields the memory.
\end{itemize}

These unify soft symmetry, flux conservation, and holography under a
single geometric structure.

% ---------------------------------------------------------------
\subsection{Interior Access and Reconstruction (Law XII)}

Law~XII identifies a boundary-defined region---the \emph{Temporal Causal
Wedge}---inside which interior observables are reconstructible from
temporal soft data.  
This provides the bridge to Pillar~II, where black-hole interior
structure is explored.

% ---------------------------------------------------------------
\subsection{Overall Structure}

The twelve Laws thus assemble into the following architecture:

\[
\boxed{
\begin{aligned}
&\text{Constraint} + \text{Evolution} \\
&\qquad\Downarrow \\
&\text{Flux} + \text{Soft Algebra} + \text{Phase Space} \\
&\qquad\Downarrow \\
&\text{Entropy} + \text{Holography} + \text{Variational Principle} \\
&\qquad\Downarrow \\
&\text{Interior Reconstruction}
\end{aligned}
}
\]


Every component is a consequence of the same derived temporal connection,
with no extension of GR’s degrees of freedom.

% ================================================================


% ================================================================
\section{Discussion: Physical Interpretation and Conceptual Significance}
\label{sec:discussion}

The temporal sector developed in this work represents a reorganisation
of general relativity’s infrared and kinematical structure, rather than a
modification of the theory.  
Its central object, the temporal connection
\(
\A_\mu = (2\alpha)^{-1}\omega_{\nu\mu}U^\nu,
\)
is a derived functional of the observer congruence and the spacetime
metric.  
The twelve Laws of the Classical Temporal Sector therefore provide
interpretive and structural clarity without altering the content of GR.

This section summarises the conceptual role of temporal geometry and the
significance of each structural component.

% ---------------------------------------------------------------
\subsection{Operational time transport and vorticity}

In general relativity, ``time'' is defined operationally by the
accumulated proper time along a worldline.  
Two observers following different timelike curves do not generally agree
on elapsed time, and transporting a clock standard around a closed loop
leads to desynchronisation governed by the vorticity tensor
$\omega_{\mu\nu}$.

The temporal connection encodes this desynchronisation in a covariant,
gauge-independent manner.  
Its holonomy measures the mismatch between transported proper times.
This elevates vorticity from a local property of a congruence to a
boundary-accessible quantity with global implications.

% ---------------------------------------------------------------
\subsection{Temporal curvature as a measure of non-integrability}

The temporal curvature $\F_{\mu\nu}=2\nabla_{[\mu}\A_{\nu]}$ measures the 
failure of $\A_\mu$ to be integrable, i.e.\ the failure of global
synchronisation to exist.

Although $\F_{\mu\nu}$ is constructed entirely from $(g_{\mu\nu},U^\mu)$,
its projection to $\mathscr I^+$ determines soft charges, memory, and
holographic kernels.  
Thus, the curvature of temporal transport carries physical information
restricted not to matter fields or propagating modes, but to the
nonlocal structure of time itself.

% ---------------------------------------------------------------
\subsection{The IR temporal mode: a reorganisation of the Bondi shear}

One of the central results is that the infrared temporal mode
\[
\Atemp_A = -\sigma^0\bar m_A - \bar\sigma^0 m_A
\]
is a spin–1 projection of the Bondi shear.  
It contains no new degrees of freedom, yet it isolates the part of the
shear that governs temporal memory and temporal soft symmetry.

This projection disentangles time-related information from the full
spin–2 gravitational wave content, enabling a focused analysis of how
time is encoded in asymptotic geometry.

% ---------------------------------------------------------------
\subsection{Soft symmetry and canonical structure}

The temporal soft transformation
\(
\delta_\Lambda A_A = D_A\Lambda
\)
is a redundancy among IR gravitational vacua.
It leaves the Bondi news invariant and acts only on the memory sector.

Laws~V and VI show that this symmetry is canonical, generated by a
Wald--Zoupas charge on the inherited symplectic structure.  
Thus the temporal soft symmetry is not a new gauge symmetry: it is the
spin–1 shadow of known large-gauge transformations in GR.

This canonical structure plays a dual role:
\begin{itemize}
\item It provides observables that are sensitive to memory.
\item It identifies directions in phase space along which the symplectic
form is degenerate (pure IR shifts).
\end{itemize}

% ---------------------------------------------------------------
\subsection{Entropic and modular aspects}

Laws~VII and VIII reveal a modular structure in the temporal IR sector.
Although $K_{\rm T}$ is not a physical Hamiltonian, its flow reorganises
IR data and defines a monotonic entropy functional.

This leads to an ``intrinsic arrow of time’’:
\[
\frac{d}{d\tau} S_{\rm temp} \ge 0,
\qquad
\frac{d}{d\tau} S_{\rm rel} \le 0,
\]
mirroring properties of relative entropy in algebraic QFT.

This emergence of temporal directionality is noteworthy:
it does not arise from thermodynamics, matter interactions, or quantum
effects—only from the structure of classical GR’s infrared shear.

% ---------------------------------------------------------------
\subsection{Holography without new fields}

Laws~IX–XI establish a boundary-to-bulk dictionary for the temporal
sector.  
Bulk temporal curvature can be reconstructed from temporal soft fields
via retarded kernels, and soft charges correspond to fluxes derived from
the Bianchi identity.

This provides a ``soft holography’’ that is entirely classical and GR-based:
\begin{itemize}
\item bulk information accessible from $\mathscr I^+$ is encoded in the
temporal shear projection,
\item soft charge conservation is equivalent to a curvature identity,
\item the memory effect is the Hamilton--Jacobi variation of a boundary
functional.
\end{itemize}

This holography reflects the causal and geometric role of temporal
transport: only regions where the temporal flow remains regular are
reconstructible.

% ---------------------------------------------------------------
\subsection{Temporal Causal Wedge}

The final Law identifies the subset of the black-hole interior that is
accessible via temporal soft data.

The resulting Temporal Causal Wedge:
\[
\mathcal{W}_{\rm TC}
=
J^{-}(\gamma_{\rm temp})
\cap
\mathcal{M}_{\rm BH},
\]
is a causally-defined region determined not by null geodesics or trapped
surfaces, but by the behaviour of the derived temporal congruence.

This structure naturally leads into Pillar~II, where interior
reconstruction and the fate of temporal curvature near horizons are
analysed.

% ---------------------------------------------------------------
\subsection{Conceptual synthesis}

The temporal sector reveals several deep principles:
\begin{itemize}
\item \textbf{Time is geometric and relational}: it derives from congruence
kinematics and curvature, not from a fundamental field.
\item \textbf{Temporal information is holographic}: asymptotic shear
encodes all causally accessible temporal curvature.
\item \textbf{Temporal structure is inherently infrared}: the entire
sector emerges from zero-frequency gravitational data.
\item \textbf{The arrow of time is encoded in the IR}: monotonic
functionals arise from canonical flows, without invoking quantum effects.
\end{itemize}

These insights place the classical temporal sector on the same footing
as the well-studied BMS structure of gravitational radiation, but with a
distinct physical interpretation tied to operational time transport.

% ================================================================


% ================================================================
\section{Conclusion}
\label{sec:conclusion}

This work has presented a complete and self-contained formulation of the
\emph{Classical Temporal Sector} of general relativity.  
The temporal connection 
\(
\A_\mu = (2\alpha)^{-1}\omega_{\nu\mu}U^\nu
\),
derived from vorticity and congruence kinematics, provides a covariant
encoding of proper-time transport without introducing any new degrees of
freedom.  
The twelve Temporal Laws developed here reveal that temporal geometry
forms a closed, holographically accessible, canonically organised
subsystem of general relativity.

The principal achievements of this Pillar are:

\begin{itemize}
\item Establishing a \textbf{constraint–evolution} structure  
      (Laws~I–II) analogous to the Hamiltonian and Raychaudhuri
      equations, but acting on the derived temporal connection.

\item Demonstrating \textbf{conservation and flux laws} (Laws~III and X)
      grounded in the Bondi expansion and the Bianchi identity, showing
      that temporal soft charges and memory are necessary features of
      asymptotically-flat gravity.

\item Identifying the \textbf{temporal soft symmetry} (Law~IV) and its
      \textbf{canonical algebra} (Laws~V–VI), inherited directly from the
      gravitational radiative phase space.

\item Constructing the \textbf{modular and entropic structure}
      (Laws~VII–VIII), revealing a monotonic ``IR arrow of time’’
      intrinsic to the spin–1 projection of the shear.

\item Establishing a complete \textbf{soft holographic dictionary}
      (Law~IX) enabling causal boundary reconstruction of temporal
      curvature in the bulk, without new fields.

\item Showing that temporal memory arises as a \textbf{Hamilton--Jacobi
      variation} of a boundary functional (Law~XI), unifying symmetry,
      flux, and variational principles.

\item Introducing the \textbf{Temporal Causal Wedge} (Law~XII), the
      region of a black-hole interior that is reconstructible from
      temporal soft data at null infinity.
\end{itemize}

Together, these results demonstrate that the temporal sector provides a
precise and powerful organisational framework for properties of time in
general relativity.  
Although the sector introduces no new dynamics, it exposes geometric,
holographic, and entropic features that remain hidden in the standard
$1{+}3$ and Bondi formulations.

\paragraph{Outlook toward Pillar~II.}
The Temporal Causal Wedge introduced here is the natural bridge to 
the next stage of the Temporal Rasa Compendium.  
Pillar~II extends these ideas to black-hole interiors, where the derived
temporal curvature interacts with horizon geometry, soft hair, memory,
and interior reconstruction.  
The twelve laws of the Classical Temporal Sector form the indispensable
foundation for these developments.

\begin{quote}
\emph{
The temporal sector is not an extension of GR.  
It is GR, reorganised to reveal the deep geometric structure underlying
operational time, infrared symmetry, and causal reconstruction.
}
\end{quote}

% ================================================================


% ================================================================
\begin{thebibliography}{99}

\bibitem{PaperI}
Shivaraj~S.~Galagali,
``Operational Time Geometry in General Relativity,'' (2025).
\href{https://doi.org/10.5281/zenodo.17813825}{doi:10.5281/zenodo.17813825}.

\bibitem{PaperII}
Shivaraj~S.~Galagali,
``Operational Reconstruction of Local Spacetime Geometry,'' (2025).
\href{https://doi.org/10.5281/zenodo.17833292}{doi:10.5281/zenodo.17833292}.

\bibitem{PaperIII}
Shivaraj~S.~Galagali,
``Newman--Penrose Formulation of the Temporal Connection,'' (2025).
\href{https://doi.org/10.5281/zenodo.17842271}{doi:10.5281/zenodo.17842271}.

\bibitem{PaperIV}
Shivaraj~S.~Galagali,
``Temporal Soft Charges and Memory in Asymptotically Flat Gravity,'' (2025).
\href{https://doi.org/10.5281/zenodo.17849331}{doi:10.5281/zenodo.17849331}.

\bibitem{PaperV}
Shivaraj~S.~Galagali,
``The Ultraviolet Temporal Sector of General Relativity'' (2025).
\href{https://doi.org/10.5281/zenodo.17859446}{doi:10.5281/zenodo.17859446}.

\bibitem{Bondi1962}
H.~Bondi, M.~G.~J.~van der Burg, and A.~W.~K.~Metzner,
``Gravitational waves in general relativity VII. Waves from isolated axisymmetric systems,''
\emph{Proc. R. Soc. Lond. A} \textbf{269}, 21 (1962).

\bibitem{Sachs1962}
R.~Sachs,
``Gravitational waves in general relativity VIII. Waves in asymptotically flat space-times,''
\emph{Proc. R. Soc. Lond. A} \textbf{270}, 103 (1962).

\bibitem{Strominger2014}
A.~Strominger,
``On BMS invariance of gravitational scattering,''
\emph{JHEP} \textbf{07}, 152 (2014),
arXiv:1312.2229 [hep-th].

\bibitem{WaldZoupas2000}
R.~M.~Wald and A.~Zoupas,
``A general definition of conserved quantities in general relativity and other theories of gravity,''
\emph{Phys. Rev. D} \textbf{61}, 084027 (2000).

\bibitem{Raychaudhuri1955}
A.~Raychaudhuri,
``Relativistic cosmology. I,''
\emph{Phys. Rev.} \textbf{98}, 1123 (1955).

\bibitem{Ashtekar1981}
A.~Ashtekar,
\emph{Asymptotic Quantization},
Monographs and Textbooks in Physical Science, Bibliopolis (1987).

\bibitem{NewmanPenrose1962}
E.~Newman and R.~Penrose,
``An approach to gravitational radiation by a method of spin coefficients,''
\emph{J. Math. Phys.} \textbf{3}, 566 (1962).

\bibitem{HollandsIshibashiWald}
S.~Hollands, A.~Ishibashi, and R.~M.~Wald,
``BMS supertranslations and memory in four and higher dimensions,''
\emph{Class. Quant. Grav.} \textbf{34}, 155005 (2017).

\end{thebibliography}
% ================================================================

\end{document}


