
\pdfoutput=1
\documentclass[11pt]{article}

% --------------------------------------------------------------
% PACKAGES (ARXIV-SAFE)
% --------------------------------------------------------------
\usepackage{amsmath,amssymb,amsfonts,bm,mathrsfs,mathtools}
\usepackage{geometry}
\usepackage{authblk}
\usepackage{tensor}
\usepackage{cite}
\usepackage{hyperref}
\usepackage{microtype}

\geometry{margin=1in}
\numberwithin{equation}{section}

\hypersetup{
    colorlinks=true,
    citecolor=blue,
    linkcolor=blue,
    urlcolor=blue
}

% --------------------------------------------------------------
% GLOBAL MACROS
% --------------------------------------------------------------
\newcommand{\Umu}{U^\mu}
\newcommand{\nmu}{n^\mu}
\newcommand{\al}{\alpha}
\newcommand{\vort}{\omega_{\mu\nu}}
\newcommand{\shear}{\sigma_{\mu\nu}}
\newcommand{\expn}{\theta}
\newcommand{\hproj}{h_{\mu\nu}}

% theorem envs
\newtheorem{theorem}{Theorem}[section]
\newtheorem{proposition}[theorem]{Proposition}
\newtheorem{definition}[theorem]{Definition}
\newtheorem{lemma}[theorem]{Lemma}

% --------------------------------------------------------------
% TITLE
% --------------------------------------------------------------
\title{\bf
Temporal Soft Charges and Memory in Asymptotically Flat Gravity:\\
Operational Infrared Structure from Clock Holonomy\\[4pt]
\large (Paper IV of the Temporal Rasa Series)
}

\author[1]{Shivaraj S.~Galagali}
\affil[1]{Independent Researcher\\
\texttt{shivarajsgalagali5@gmail.com}}
\date{\today}

\begin{document}
\maketitle

% --------------------------------------------------------------
\begin{abstract}
Papers~I–III developed an operational description of relativistic time
based on a timelike reference congruence $n^\mu$, a physical observer
$U^\mu$, and the measurable scalar $\alpha=-n_\mu U^\mu$.
Non-integrability of clock transport is encoded in the vorticity
$\omega_{\mu\nu}$ of $n^\mu$.

Here we analyse the infrared structure of this non-integrability at
future null infinity $\mathscr I^+$.  Using the covariant temporal
connection
\[
\mathcal A_\mu = \frac{1}{2\alpha}({\star_U}\omega)_\mu,
\qquad
({\star_U}\omega)_\mu
= \tfrac12 \varepsilon_{\mu\nu\rho\sigma}U^\nu \omega^{\rho\sigma},
\]
we show that its radiative mode at $\mathscr I^+$ is fixed completely by
the Bondi shear~$\sigma^0$:
\[
\boxed{
\mathcal A_A^{(0)}
= \frac14\!\left(\sigma^0 m_A+\bar\sigma^0\bar m_A\right).
}
\]
This is a spin-weight $+1$ projection of gravitational radiation and
contains no new degrees of freedom.

We define the associated soft symmetry, Wald--Zoupas charge, and the
resulting ``temporal holonomy memory''.  Only the electric-parity
component of $\sigma^0$ contributes to the permanent effect.  These
structures establish the IR completion of the operational time geometry
developed in Papers~I–III.
\end{abstract}

\tableofcontents
\newpage

% ===============================================================
\section{Introduction}

Papers~I–III introduced an operational time geometry based on:
\begin{itemize}
\item a reference congruence $n^\mu$,
\item a physical observer $U^\mu$,
\item the observable lapse $\al=-n_\mu U^\mu$,
\item and the vorticity two-form $\omega_{\mu\nu}$ of $n^\mu$.
\end{itemize}

Clock transport is path dependent whenever $\omega_{\mu\nu}\neq 0$.
The resulting temporal connection,
\[
\mathcal A_\mu = \frac{1}{2\alpha}({\star_U}\omega)_\mu,
\]
encodes this non-integrability.

In asymptotically flat spacetimes, the IR limit of this structure
projects onto the Bondi radiative data.  Our goals:

\begin{enumerate}
\item Identify the radiative mode $\mathcal A_A^{(0)}$ at $\mathscr I^+$.
\item Show it is determined entirely by the shear~$\sigma^0$.
\item Define the associated soft symmetry and charge.
\item Derive a ``temporal memory'' in clock holonomy.
\end{enumerate}

No new fields are introduced; all IR data descend from the gravitational
shear.

% ===============================================================
\section{Temporal Connection and Holonomy}

Let $n^\mu$ be a unit timelike congruence and $U^\mu$ a physical
observer.  The vorticity of $n^\mu$ is
\[
\omega_{\mu\nu}
= h_\mu{}^\rho h_\nu{}^\sigma \nabla_{[\rho}n_{\sigma]},
\qquad
h_{\mu\nu}=g_{\mu\nu}+n_\mu n_\nu.
\]

The spatial Hodge dual relative to $U^\mu$ is
\[
({\star_U}\omega)_\mu
= \tfrac12 \varepsilon_{\mu\nu\rho\sigma}U^\nu\omega^{\rho\sigma}.
\]

\begin{definition}[Temporal connection]
\[
\boxed{
\mathcal A_\mu = \frac{1}{2\al}({\star_U}\omega)_\mu .
}
\]
\end{definition}

For any small loop $C$ in the rest frame of~$U^\mu$,
\[
\oint_C \mathcal A_\mu dx^\mu
= \int_\Sigma \omega_{\mu\nu} dS^{\mu\nu},
\]
the vorticity flux first identified in Paper~I.

% ===============================================================
\section{Asymptotic Structure at $\mathscr I^+$}

Bondi coordinates $(u,r,x^A)$ yield
\[
\sigma = \frac{\sigma^0(u,x^A)}{r} + O(r^{-2}),
\qquad
m_A = m_\mu \partial_A x^\mu.
\]

A physically natural asymptotic observer satisfies
\[
U^\mu = \frac{\ell^\mu+N^\mu}{\sqrt2} + O(r^{-1}),
\qquad
\al=1+O(r^{-1}),
\]
and $n^\mu$ aligns with~$\ell^\mu$.

Projecting $\mathcal A_\mu$ to the sphere:
\[
\mathcal A_A = \mathcal A_\mu\partial_A x^\mu
= \frac{1}{r}\mathcal A_A^{(0)} + O(r^{-2}).
\]

\begin{theorem}[Temporal--shear correspondence]
\[
\boxed{
\mathcal A_A^{(0)}
= \frac14(\sigma^0 m_A + \bar\sigma^0 \bar m_A).
}
\]
\end{theorem}

This is a spin-weight $+1$ field determined by shear alone.

\paragraph{Properties.}
\begin{itemize}
\item No Coulombic ($\ell=0,1$) data contribute.
\item No new state-dependent function enters; IR data is inherited from $\sigma^0$.
\item Under $m_A\to e^{i\chi}m_A$, the field transforms with spin~$+1$.
\end{itemize}

% ===============================================================
\section{Electric and Magnetic Parity}

Decompose
\[
\sigma^0 = \sigma_E^0 + i\sigma_B^0.
\]

In a real dyad basis $\{e_A^{(1)},e_A^{(2)}\}$ defined through
$m_A = \tfrac{1}{\sqrt2}(e_A^{(1)}+i e_A^{(2)})$, the radiative mode becomes:
\[
\mathcal A_A^{(0)}
= \frac12 \big[\sigma_E^0\, e_A^{(1)} - \sigma_B^0\, e_A^{(2)}\big].
\]

Thus:
\[
\boxed{
\mathcal A_A^E \propto \sigma_E^0,
\qquad
\mathcal A_A^B \propto \sigma_B^0 .
}
\]

Only $\sigma_E^0$ contributes to the permanent memory.

% ===============================================================
\section{Soft Symmetry and Charge}

At $\mathscr I^+$, define the operational symmetry
\[
\delta_\Lambda \mathcal A_A^{(0)} = 0,
\]
(no radiative gauge term appears at leading order), and the charge:
\[
\boxed{
Q_\Lambda
= \frac{1}{32\pi}
\int_{S^2} \Lambda\, D^A\!\left(\sigma^0 m_A+\bar\sigma^0\bar m_A\right)\! d\Omega .
}
\]

Only $\ell\ge2$ modes contribute.  
The algebra is abelian:
\[
\{Q_\Lambda,Q_{\Lambda'}\}=0,
\qquad
\{Q_\Lambda,Q_{\xi}^{\rm BMS}\}=0.
\]

% ===============================================================
\section{Temporal Memory}

Since $N=\partial_u\sigma^0$,
\[
\partial_u\mathcal A_A^{(0)}
= \frac14(N m_A+\bar N\bar m_A).
\]

Integrating:
\[
\Delta\mathcal A_A^{(0)}
= \frac14(\Delta\sigma^0 m_A+\Delta\bar\sigma^0\bar m_A).
\]

The change in a clock holonomy around a loop $C$ is:
\[
\boxed{
\Delta \ln\mathcal H[C]
= \oint_C \Delta\mathcal A_A^{(0)} dx^A .
}
\]

If $\Delta\mathcal A_A^{(0)}=\partial_A f$ (pure gauge), the integral
vanishes. The permanent temporal memory arises only from the electric
parity of $\sigma^0$, exactly paralleling E-mode displacement memory.

% ===============================================================
\section{Observer-Class Invariance}

If $\tilde U^\mu$ is any asymptotically inertial observer:
\[
\tilde U^\mu = U^\mu + O(r^{-1}),
\qquad
U_\mu\tilde U^\mu=-1+O(r^{-1}),
\]
then
\[
\tilde{\mathcal A}_A^{(0)} = \mathcal A_A^{(0)}.
\]

Thus the temporal IR mode is invariant under the standard Bondi class of
asymptotic inertial observers.

% ===============================================================
\section{Conclusion}

The operational temporal connection of Papers~I–III acquires a clean
infrared limit at future null infinity.  The radiative mode
\[
\mathcal A_A^{(0)}
= \frac14(\sigma^0 m_A+\bar\sigma^0\bar m_A)
\]
encodes a spin-weight~$+1$ projection of gravitational radiation, defines
a consistent abelian soft charge, and generates a permanent temporal
memory proportional to the E-mode shear.

The temporal IR sector contains no new dynamical degrees of freedom: it
is a physically operational projection of the Bondi shear.  This prepares
the ground for the temporal information channel and soft entanglement
developed in Papers~V–VIII.

% ===============================================================
\appendix
\section{Derivation of the Asymptotic Formula}

The Bondi--Sachs NP tetrad satisfies
\[
\ell^\mu N_\mu=-1,\qquad m^\mu\bar m_\mu=1.
\]

At large $r$:
\[
\ell^\mu=\partial_r+O(r^{-1}),\qquad
m^\mu=\frac{1}{\sqrt2 r}\hat m^A\partial_A+O(r^{-2}),
\]
\[
\sigma = \frac{\sigma^0}{r}+O(r^{-2}).
\]

NP torsion identities yield
\[
\nabla_{[\mu}\ell_{\nu]}
=
-\sigma\,m_{[\mu}m_{\nu]}
-\bar\sigma\,\bar m_{[\mu}\bar m_{\nu]}
+O(r^{-2}).
\]

Since $n^\mu\sim \ell^\mu$,
\[
\omega_{\mu\nu}
=
-\frac{1}{r}
\left(\sigma^0 m_{[\mu}m_{\nu]}
+\bar\sigma^0\bar m_{[\mu}\bar m_{\nu]}\right)+O(r^{-2}).
\]

Contract with $U^\nu = (\ell^\nu+N^\nu)/\sqrt2$:
\[
\omega_{\nu A}U^\nu
=
\frac{1}{2r}
(\sigma^0 m_A + \bar\sigma^0\bar m_A) + O(r^{-2}).
\]

Since $\al=1+O(r^{-1})$,
\[
\mathcal A_A
= \frac{1}{2\al}\omega_{\nu A}U^\nu
= \frac{1}{4r}(\sigma^0 m_A+\bar\sigma^0\bar m_A)
+O(r^{-2}),
\]
giving the stated result.

% ===============================================================
\begin{thebibliography}{99}

\bibitem{PaperI}
Shivaraj~S.~Galagali,
``Operational Time Geometry in General Relativity,'' (2025).
\href{https://doi.org/10.5281/zenodo.17813825}{doi:10.5281/zenodo.17813825}.

\bibitem{PaperII}
Shivaraj~S.~Galagali,
``Operational Reconstruction of Local Spacetime Geometry,'' (2025).
\href{https://doi.org/10.5281/zenodo.17833292}{doi:10.5281/zenodo.17833292}.

\bibitem{PaperIII}
Shivaraj~S.~Galagali,
``Newman--Penrose Formulation of the Temporal Connection,'' (2025).
\href{https://doi.org/10.5281/zenodo.17842271}{doi:10.5281/zenodo.17842271}.

\bibitem{Bondi1962}
H.~Bondi, M.~van~der~Burg, A.~Metzner,
Proc.\ Roy.\ Soc.\ A \textbf{269}, 21 (1962).

\bibitem{Sachs1962}
R.~Sachs,
Proc.\ Roy.\ Soc.\ A \textbf{270}, 103 (1962).

\bibitem{NP}
E.~T.~Newman, R.~Penrose,
J.\ Math.\ Phys.\ \textbf{3}, 566 (1962).

\bibitem{StromingerLectures}
A.~Strominger,
\emph{Lectures on the Infrared Structure of Gravity},
Princeton University Press (2018).

\bibitem{WaldZoupas}
R.~Wald, A.~Zoupas,
Phys.\ Rev.\ D \textbf{61}, 084027 (2000).

\end{thebibliography}

\end{document}
