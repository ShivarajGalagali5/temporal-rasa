\pdfoutput=1
\documentclass[11pt]{article}

% ---------------------------------------------------------
% Packages (arXiv-safe)
% ---------------------------------------------------------
\usepackage{amsmath,amssymb,amsfonts,mathtools,bm}
\usepackage{geometry}
\usepackage{authblk}
\usepackage{tensor}
\usepackage{cite}
\usepackage{amsthm}
\usepackage{microtype}
\usepackage{hyperref}

\geometry{margin=1in}
\numberwithin{equation}{section}

\hypersetup{
  colorlinks=true,
  linkcolor=blue,
  citecolor=blue,
  urlcolor=blue
}

% ---------------------------------------------------------
% Theorems
% ---------------------------------------------------------
\newtheorem{theorem}{Theorem}[section]
\newtheorem{definition}[theorem]{Definition}
\newtheorem{proposition}[theorem]{Proposition}
\newtheorem{lemma}[theorem]{Lemma}

% ---------------------------------------------------------
% Macros
% ---------------------------------------------------------
\newcommand{\nmu}{n^\mu}
\newcommand{\nmd}{n_\mu}
\newcommand{\Umu}{U^\mu}
\newcommand{\Umd}{U_\mu}

\newcommand{\an}{a_\mu^{(n)}}
\newcommand{\aU}{a_U^\mu}

\newcommand{\hproj}{h_{\mu\nu}}

\newcommand{\shear}{\sigma_{\mu\nu}}
\newcommand{\vort}{\omega_{\mu\nu}}
\newcommand{\expn}{\theta}

\newcommand{\Kgen}{K_{\mu\nu}}

\newcommand{\kU}{(k\!\cdot\!U)}
\newcommand{\kn}{(k\!\cdot\!n)}

% ---------------------------------------------------------
% Title
% ---------------------------------------------------------
\title{\bf
Operational Time Geometry in General Relativity:\\
The $\alpha$--Framework and Exact Observer Kinematics\\[4pt]
{\large (Paper I of the Temporal Rasa Series)}
}

\author[1]{Shivaraj S.~Galagali}
\affil[1]{Independent Researcher\\
\texttt{shivarajsgalagali5@gmail.com}}
\date{\today}

\begin{document}
\maketitle

\begin{abstract}
We develop a covariant and fully operational formulation of relativistic
time based on the measurable scalar
$\alpha = -n_\mu U^\mu$.  Physically,
$\alpha$ equals the Lorentz factor of a physical observer $U^\mu$
relative to a reference congruence $n^\mu$, i.e.
$\alpha = d\tau_n/d\tau_U$.
No hypersurface-orthogonality assumptions are made.

We derive: (i) the exact master evolution equation for $\alpha$,
(ii) an exact second evolution identity for $\ddot{\alpha}$,
(iii) the fully corrected null frequency-transport equation including
reference-frame acceleration terms, and (iv) a temporal holonomy
expression identifying vorticity as the principal obstruction to global
synchronization.

These identities form the kinematic backbone of an operational geometry
of time that applies to rotating congruences, gravitating systems, and
arbitrary observers.
\end{abstract}

\tableofcontents
\newpage

% =========================================================
\section{Introduction}

Time measurements in general relativity are operational: clocks compare
proper times, and photon frequencies encode relative motion and gravity.
To express these measurements covariantly and independently of
hypersurface-orthogonality, we introduce the scalar
\begin{equation}
\boxed{\alpha = -n_\mu U^\mu},
\label{eq:alphadef}
\end{equation}
defined for any smooth timelike congruence $n^\mu$ with
$n_\mu n^\mu = -1$ and any physical observer $U^\mu$.

In the instantaneous rest frame of $n^\mu$,
\[
n^\mu=(1,0,0,0),\qquad
U^\mu = \gamma(1,\mathbf{v}),
\]
so that
\[
\alpha = -n_\mu U^\mu = \gamma = \frac{d\tau_n}{d\tau_U}.
\]
Thus $\alpha$ measures how much \emph{faster} the reference congruence's
proper time accumulates relative to the observer's.

The purpose of this paper is to derive exact evolution laws for
$\alpha$, for photons interacting with observers, and for operational
time holonomy.

% =========================================================
\section{Congruence Kinematics}

We adopt signature $(-,+,+,+)$.

A smooth timelike congruence $n^\mu$ satisfies $n_\mu n^\mu=-1$ and
admits the standard decomposition
\begin{equation}
\nabla_\mu n_\nu
=
- n_\mu \an
+ \shear
+ \vort
+ \tfrac13 \theta\,\hproj,
\label{eq:decomp}
\end{equation}
where $\hproj = g_{\mu\nu}+n_\mu n_\nu$ is the spatial projector.

A general observer decomposes relative to $n^\mu$ as
\begin{equation}
U^\mu = \alpha n^\mu + v^\mu,\qquad
n_\mu v^\mu=0,\qquad
v_\mu v^\mu=\alpha^2-1.
\label{eq:Usplit}
\end{equation}
We define $\dot X = U^\rho\nabla_\rho X$.

Let
\[
K_{\mu\nu}=\sigma_{\mu\nu}+\tfrac13\theta h_{\mu\nu}
\]
denote the spatial symmetric part of $\nabla_\mu n_\nu$.  (It agrees
with extrinsic curvature only in the hypersurface-orthogonal case.)

% =========================================================
\section{Master Evolution Equation}

\begin{theorem}[Master equation]
\label{thm:master}
\begin{equation}
\boxed{
\dot{\alpha}
=
-\alpha\, a^{(n)}_\mu U^\mu
- K_{\mu\nu}U^\mu U^\nu
- n_\mu a_U^\mu,
}
\label{eq:master}
\end{equation}
where $a_U^\mu := U^\rho\nabla_\rho U^\mu$.
\end{theorem}

Proof: Appendix~\ref{app:master}.

% =========================================================
\section{Second Evolution Identity for \texorpdfstring{$\alpha$}{alpha}}

Define
\[
A := \alpha\, a^{(n)}_\mu U^\mu,\qquad
B := K_{\mu\nu}U^\mu U^\nu,\qquad
C := n_\mu a_U^\mu.
\]

\begin{theorem}
\label{thm:alpharay}
\begin{equation}
\boxed{
\ddot{\alpha}
=
- U^\rho\nabla_\rho A
- U^\rho\nabla_\rho B
- U^\rho\nabla_\rho C.
}
\label{eq:alpharay}
\end{equation}
\end{theorem}

This identity follows directly from \eqref{eq:master}.  It is an exact
second evolution equation for $\alpha$; no curvature decomposition is
implied.

For $U^\mu=n^\mu$, one has $\alpha=1$ and $A=B=C=0$, hence
$\ddot{\alpha}=0$ as required.

% =========================================================
\section{Photon Frequency Transport}

Let $k^\mu$ be a null geodesic tangent, affinely parametrized by
$\lambda$, and let an observer measure the frequency
\[
\nu = -k_\mu U^\mu.
\]

\begin{theorem}[Operational null frequency transport]
\label{thm:nulltransport}
\begin{equation}
\boxed{
\begin{aligned}
\frac{d}{d\lambda}\!\left(\frac{\nu}{\alpha}\right)
=&\;
(\kn)(k\!\cdot\!a^{(n)})
- k^\rho k^\mu \sigma_{\rho\mu}
- \tfrac13 \theta\,(\kn)^2
\\[4pt]
&\;
- \frac{1}{\alpha} \, k^\rho k^\mu \nabla_\rho v_\mu
+ \frac{k\!\cdot\!v}{\alpha^2}\,k^\rho\nabla_\rho\alpha.
\end{aligned}
}
\label{eq:nulltransport}
\end{equation}
\end{theorem}

This is the exact, fully corrected operational redshift law including
all shear, expansion, reference-acceleration, and Doppler-gradient
effects.

% =========================================================
\section{Temporal Holonomy}

Define the time 1-form $\tau := n_\mu dx^\mu$.  Its exterior derivative
is
\begin{equation}
(d\tau)_{\mu\nu}
= 2\nabla_{[\mu}n_{\nu]}
= 2(\omega_{\mu\nu}-n_{[\mu}a^{(n)}_{\nu]}).
\end{equation}

\begin{theorem}[Vorticity obstruction to synchronization]
\label{thm:holonomy}
\begin{equation}
\boxed{
\oint_\gamma n_\mu dx^\mu
=
2\iint_\Sigma \omega_{\mu\nu}\, dS^{\mu\nu}
-
2\iint_\Sigma n_{[\mu}a^{(n)}_{\nu]}\, dS^{\mu\nu}.
}
\label{eq:hol}
\end{equation}
\end{theorem}

Vorticity gives the leading obstruction to integrable simultaneity.

% =========================================================
\section{Hypersurface-Orthogonal (ADM) Limit}

If $n^\mu$ is hypersurface-orthogonal to $t=\mathrm{const}$ slices and
\[
n_\mu = -N\nabla_\mu t,
\]
then for any $U^\mu$,
\[
\alpha = -n_\mu U^\mu = N\,U^0.
\]
Thus $\alpha$ reduces to the ADM lapse $N$ \emph{only} for Eulerian
observers with $U^\mu=n^\mu$.

% =========================================================
\section{Conclusion}

The scalar $\alpha=-n_\mu U^\mu$ provides a clean operational measure of
relative clock rates in general spacetimes.  We derived exact evolution
laws for $\alpha$ and its derivatives, a fully corrected photon
frequency-transport law, and a temporal holonomy formula.

These structures require no foliation or hypersurface-orthogonality and
apply equally to rotating congruences and gravitating systems.  They
form the kinematic basis for a more general operational time geometry
developed in subsequent work.

% =========================================================
\appendix

\section{Derivation of the Master Equation}
\label{app:master}

From $\alpha=-n_\mu U^\mu$,
\[
\dot\alpha
=
-U^\rho\nabla_\rho(n_\mu U^\mu)
=
-U^\rho U^\mu\nabla_\rho n_\mu
- n_\mu a_U^\mu.
\]

Insert \eqref{eq:decomp}, use $U^\mu U^\nu\omega_{\mu\nu}=0$, and rewrite
\[
U^\rho U^\mu\nabla_\rho n_\mu
=
\alpha a^{(n)}_\mu U^\mu
+ K_{\mu\nu}U^\mu U^\nu.
\]
This yields \eqref{eq:master}.  \qed

% =========================================================
\section{Null Transport Derivation}

Start with $k^\rho\nabla_\rho(k_\mu U^\mu)=0$ and use the decomposition
$U_\mu=\alpha n_\mu+v_\mu$, together with $k^\mu k^\nu\vort_{\mu\nu}=0$.
A short calculation yields \eqref{eq:nulltransport}.  \qed

% =========================================================

\begin{thebibliography}{99}

% Foundations and textbooks
\bibitem{Wald}
R.~M.~Wald,
\textit{General Relativity},
University of Chicago Press (1984).

\bibitem{MTW}
C.~W.~Misner, K.~S.~Thorne, and J.~A.~Wheeler,
\textit{Gravitation},
W.~H.~Freeman (1973).

\bibitem{Carroll}
S.~Carroll,
\textit{Spacetime and Geometry},
Addison-Wesley (2004).

\bibitem{PoissonToolkit}
E.~Poisson,
\textit{A Relativist's Toolkit},
Cambridge University Press (2004).

\bibitem{PoissonWill}
E.~Poisson and C.~M.~Will,
\textit{Gravity},
Cambridge University Press (2014).

% 1+3 covariant formalism and observer kinematics
\bibitem{Ehlers}
J.~Ehlers,
``Survey of General Relativity Theory'',
in \textit{Relativity, Astrophysics and Cosmology},
Springer (1971).

\bibitem{EllisElst}
G.~F.~R.~Ellis and H.~van Elst,
``Cosmological Models'',
in \textit{Cargèse Lectures}, Springer (1999).

\bibitem{Maartens}
R.~Maartens,
``Covariant Fluid Dynamics in Cosmology'',
\textit{Class.\ Quant.\ Grav.}\ \textbf{12}, 1455 (1995).

\bibitem{HawkingEllis}
S.~W.~Hawking and G.~F.~R.~Ellis,
\textit{The Large Scale Structure of Spacetime},
Cambridge University Press (1973).

% Redshift, null transport, and optical geometry
\bibitem{Synge}
J.~L.~Synge,
\textit{Relativity: The General Theory},
North-Holland (1960).

\bibitem{FaraoniRedshift}
V.~Faraoni,
``Cosmological Redshift: Confusion Over Interpretation'',
\textit{Am.\ J.\ Phys.}\ \textbf{67}, 732 (1999).

\bibitem{Bardeen}
J.~M.~Bardeen,
``Gauge-Invariant Cosmological Perturbations'',
\textit{Phys.\ Rev.\ D}\ \textbf{22}, 1882 (1980).

% ADM and foliation-based structures (for lapse comparisons)
\bibitem{ADM}
R.~Arnowitt, S.~Deser, and C.~W.~Misner,
``The Dynamics of General Relativity'',
\textit{Gen.\ Rel.\ Grav.}\ \textbf{40}, 1997 (2008).

\bibitem{Gourgoulhon}
E.~Gourgoulhon,
\textit{3+1 Formalism in General Relativity},
Springer (2012).

% Differential geometry and Frobenius theorem
\bibitem{ChoquetBruhat}
Y.~Choquet-Bruhat,
\textit{General Relativity and the Einstein Equations},
Oxford University Press (2009).

\end{thebibliography}


\end{document}