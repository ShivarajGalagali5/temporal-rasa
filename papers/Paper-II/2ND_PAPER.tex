\pdfoutput=1
\documentclass[11pt]{article}

% ---------------------------------------------------------
% PACKAGES
% ---------------------------------------------------------
\usepackage{amsmath,amssymb,amsfonts,bm,mathtools}
\usepackage{geometry}
\usepackage{authblk}
\usepackage{tensor}
\usepackage{cite}
\usepackage{amsthm}
\usepackage{mathrsfs}
\usepackage{hyperref}
\usepackage{microtype}
\usepackage{enumitem}

\geometry{margin=1in}
\numberwithin{equation}{section}

\hypersetup{
  colorlinks=true,
  linkcolor=blue,
  citecolor=blue,
  urlcolor=blue
}

% ---------------------------------------------------------
% THEOREMS
% ---------------------------------------------------------
\newtheorem{theorem}{Theorem}[section]
\newtheorem{proposition}[theorem]{Proposition}
\newtheorem{definition}[theorem]{Definition}
\newtheorem{lemma}[theorem]{Lemma}

% ---------------------------------------------------------
% MACROS
% ---------------------------------------------------------
\newcommand{\nmu}{n^\mu}
\newcommand{\nmd}{n_\mu}
\newcommand{\Umu}{U^\mu}
\newcommand{\Umd}{U_\mu}
\newcommand{\al}{\alpha}

\newcommand{\shear}{\sigma_{\mu\nu}}
\newcommand{\vort}{\omega_{\mu\nu}}
\newcommand{\expn}{\theta}

\newcommand{\hproj}{h_{\mu\nu}}
\newcommand{\curl}{\mathrm{curl}}
\newcommand{\STF}[1]{\langle #1\rangle}

% ---------------------------------------------------------
% TITLE
% ---------------------------------------------------------
\title{\bf
Operational Reconstruction of Local Spacetime Geometry in General Relativity:\\
Clocks, Light, and Congruence Kinematics as Complete Observables\\[4pt]
{\large (Paper II of the Temporal Rasa Series)}
}

\author[1]{Shivaraj S.~Galagali}
\affil[1]{Independent Researcher\\
\texttt{shivarajsgalagali5@gmail.com}}

\date{\today}

\begin{document}
\maketitle

\begin{abstract}
Paper~I introduced the operational lapse $\alpha=-n_\mu U^\mu$ and a
temporal connection encoding the non-integrability of observer time.
Here we show that, for any smooth physical observer congruence $U^\mu$,
the directly measurable operational data set
\[
\{\theta,\ \sigma_{\mu\nu},\ \omega_{\mu\nu},\ a_\mu;\ \rho,p,q_\mu,\pi_{\mu\nu}\}
\]
is sufficient to reconstruct the full local spacetime geometry
$(g_{\mu\nu},T_{\mu\nu})$ up to diffeomorphisms on a simply-connected
convex normal neighborhood.

The reconstruction proceeds in four steps:
(i) symmetric kinematics from photon frequency transport,
(ii) vorticity from infinitesimal temporal holonomies,
(iii) curvature from the Einstein--Bianchi system, and
(iv) metric reconstruction via Cartan structure equations.

This establishes the operational foundation for temporal gauge structure and
infrared temporal geometry developed in later papers.
\end{abstract}

\tableofcontents
\bigskip

% =========================================================
\section{Observer Kinematics and Operational Quantities}

We take signature $(-+++)$ and use a physical observer congruence
$\Umu$ with $\Umd\Umu=-1$.
The spatial projector is
\begin{equation}
h_{\mu\nu} = g_{\mu\nu} + U_\mu U_\nu.
\end{equation}
The spatial covariant derivative acting on any tensor $X_{\nu\cdots}$ is
\begin{equation}
D_\mu X_{\nu\cdots} := h_\mu{}^\rho h_\nu{}^\sigma\cdots
\nabla_\rho X_{\sigma\cdots},
\label{eq:D-def}
\end{equation}
with all free indices projected into the observer rest space.

The $1{+}3$ decomposition of $\nabla_\mu U_\nu$ is
\begin{equation}
\nabla_\mu U_\nu
= \sigma_{\mu\nu} + \omega_{\mu\nu}
+ \tfrac13\theta\, h_{\mu\nu}
- U_\mu a_\nu,
\label{eq:1p3}
\end{equation}
where $a_\mu = U^\rho\nabla_\rho U_\mu$ is locally measurable (by an
accelerometer).

We retain the reference congruence $n^\mu$ from Paper~I and the
operational lapse
\begin{equation}
\al = -n_\mu U^\mu > 0.
\label{eq:alpha-def}
\end{equation}
Operationally,
\[
\alpha = \frac{d\tau_n}{d\tau_U},
\]
the Lorentz factor of $U^\mu$ relative to $n^\mu$.

The temporal one-form is $\tau = n_\mu dx^\mu$.
The temporal connection $\mathcal A_\mu$ is defined by
\[
\mathcal A_\mu dx^\mu = \alpha^{-1}\tau.
\]

% =========================================================
\section{Photon Frequency Transport and Symmetric Kinematics}

Let $k^\mu$ be an affinely parametrized null geodesic and
$\nu=-k_\mu U^\mu$ the measured photon frequency.

Paper~I derived a general exact transport identity for $\nu/\alpha$ in
terms of $(\theta,\sigma_{\mu\nu},\omega_{\mu\nu},a^{(n)}_\mu,v^\mu)$
for an arbitrary pair of congruences $U^\mu$ and $n^\mu$.  When the
reference congruence is chosen to coincide with the physical observer,
\begin{equation}
n^\mu = U^\mu,
\qquad
\alpha = 1,
\qquad
v^\mu = 0,
\label{eq:observer-gauge}
\end{equation}
this decomposition becomes trivial and the general law reduces
consistently to the standard observer-based frequency transport:
\begin{equation}
\frac{d\nu}{d\lambda}
= -k^\mu k^\nu
\left(\sigma_{\mu\nu}+\tfrac13\theta\,h_{\mu\nu}\right).
\label{eq:transport-op}
\end{equation}
Because $k^\mu k^\nu$ is symmetric, vorticity does not contribute.

At an event $p$, write $k^\mu=\omega(U^\mu+N^\mu)$ with
$N^\mu$ spatial and $N_\mu N^\mu = 1$.
Define the directional transport coefficient
\begin{equation}
\Delta_p(N)
= -\sigma_{\mu\nu}N^\mu N^\nu - \tfrac13\theta.
\label{eq:Delta}
\end{equation}

\begin{proposition}
Full-sky directional measurements of $\Delta_p(N)$ uniquely determine
$\theta$ and $\sigma_{\mu\nu}$ at $p$.
\end{proposition}

\begin{proof}
Expand $\Delta_p(N)$ in spherical harmonics on the observer’s sky.
The $\ell=0$ mode yields $\theta$, the $\ell=2$ modes yield the five
independent components of $\sigma_{\mu\nu}$; higher harmonics vanish by
construction.
\end{proof}

% =========================================================
\section{Temporal Holonomy and Vorticity}

Temporal non-integrability is encoded in $\tau=n_\mu dx^\mu$:
\begin{equation}
d\tau
= 2\nabla_{[\mu}n_{\nu]} dx^\mu\wedge dx^\nu
= 2(\omega_{\mu\nu} - n_{[\mu}a^{(n)}_{\nu]}) dx^\mu\wedge dx^\nu.
\label{eq:dTau}
\end{equation}

The curvature of the temporal connection $\mathcal A_\mu=\alpha^{-1}n_\mu$
is
\[
F_{\mu\nu} = 2\nabla_{[\mu}\mathcal A_{\nu]}.
\]
For infinitesimal spatial 2-surfaces orthogonal to $U^\mu$, the
$vorticity$ term dominates the holonomy, and one effectively measures
$\omega_{\mu\nu}$.

For a small spatial loop $C$ enclosing a 2-surface with area bivector
$S^{\mu\nu}$ in the rest space of $U^\mu$, we have
\begin{equation}
\oint_C \mathcal A_\mu dx^\mu
= \omega_{\mu\nu} S^{\mu\nu} + O(\epsilon^3).
\label{eq:holonomy}
\end{equation}
Here $S^{\mu\nu}$ is spatial, $S^{\mu\nu}U_\nu=0$.

\begin{proposition}
Holonomies over three independent infinitesimal spatial loops
determine $\omega_{\mu\nu}$ uniquely.
\end{proposition}

\begin{proof}
Choose three independent spatial 2-planes in the observer rest space.
The three resulting holonomies give three independent linear combinations
of the components of $\omega_{\mu\nu}$, which can be inverted to obtain
$\omega_{\mu\nu}$.
\end{proof}

% =========================================================
\section{Operational Dataset}

At each event, the measured dataset is
\[
\mathcal O_{\rm op}[U] =
\{\theta,\sigma_{\mu\nu},\omega_{\mu\nu},a_\mu;\
 \rho,p,q_\mu,\pi_{\mu\nu}\}.
\]
Matter variables can be defined operationally by local stress-energy
measurements in the rest frame of $U^\mu$.

% =========================================================
\section{Einstein--Bianchi Closure and Curvature}

\subsection{Ricci Tensor}

Einstein’s equation gives
\begin{equation}
R_{\mu\nu}
= 8\pi\left(T_{\mu\nu}-\tfrac12g_{\mu\nu}T\right),
\end{equation}
so once $(\rho,p,q_\mu,\pi_{\mu\nu})$ are known, the Ricci tensor is
determined.

\subsection{Weyl Tensor}

Define the electric and magnetic parts of the Weyl tensor with respect
to $U^\mu$:
\begin{equation}
E_{\mu\nu}
= C_{\mu\alpha\nu\beta}U^\alpha U^\beta,
\qquad
B_{\mu\nu}
= \tfrac12\epsilon_{\mu\alpha\beta}
C^{\alpha\beta}{}_{\nu\gamma}U^\gamma.
\end{equation}

The spatial Bianchi constraints take the form
\begin{align}
D^\nu\sigma_{\mu\nu} - \tfrac23D_\mu\theta
+ 4\pi q_\mu
&= - E_{\mu\nu}U^\nu,
\label{eq:E-constraint}
\\
D^\nu\omega_{\mu\nu}
&= B_{\mu\nu}U^\nu.
\label{eq:B-constraint}
\end{align}
These constraints determine the longitudinal components
$E_{\mu\nu}U^\nu$ and $B_{\mu\nu}U^\nu$.  Together with symmetry,
trace-free conditions, and the evolution equations, this fixes the full
tensors $E_{\mu\nu}$ and $B_{\mu\nu}$.

The evolution equations can be written schematically as
\begin{align}
\dot{E}_{\STF{\mu\nu}}
&= -\theta E_{\mu\nu}
+ \curl B_{\mu\nu}
+ \mathcal S^{(E)}_{\mu\nu},
\\
\dot{B}_{\STF{\mu\nu}}
&= -\theta B_{\mu\nu}
- \curl E_{\mu\nu}
+ \mathcal S^{(B)}_{\mu\nu},
\end{align}
where the source terms $\mathcal S^{(E)}_{\mu\nu}$,
$\mathcal S^{(B)}_{\mu\nu}$ depend algebraically on
$(\theta,\sigma_{\mu\nu},\omega_{\mu\nu},a_\mu)$ and the matter fields.

\subsection{Full Riemann Tensor}

The Riemann tensor decomposes as
\begin{equation}
R_{\mu\nu\rho\sigma}
= C_{\mu\nu\rho\sigma}
+ g_{\mu[\rho}R_{\sigma]\nu}
- g_{\nu[\rho}R_{\sigma]\mu}
- \tfrac13R\, g_{\mu[\rho}g_{\sigma]\nu}.
\end{equation}
Thus $R_{\mu\nu\rho\sigma}$ is fully determined once $(E_{\mu\nu},B_{\mu\nu})$
and $R_{\mu\nu}$ are known.

% =========================================================
\section{Metric Reconstruction via Cartan Integration}

Choose an orthonormal frame $e^a{}_\mu$ at a point in a
simply-connected convex normal neighborhood $\mathcal U$.
The torsion-free, metric-compatible Cartan equations are
\begin{align}
de^a + \omega^a{}_b\wedge e^b &= 0,
\\
d\omega^a{}_b + \omega^a{}_c\wedge \omega^c{}_b &= R^a{}_b,
\end{align}
where $R^a{}_b$ is the curvature 2-form corresponding to
$R^\rho{}_{\sigma\mu\nu}$.

Given the curvature on $\mathcal U$, these equations uniquely determine
$(e^a{}_\mu,\omega^a{}_b)$ up to a global Lorentz transformation.
The metric is then reconstructed as
\begin{equation}
g_{\mu\nu} = \eta_{ab} e^a{}_\mu e^b{}_\nu.
\end{equation}
The remaining freedom is diffeomorphism invariance.

% =========================================================
\section{Reconstruction Theorem}

\begin{theorem}
Let $\mathcal U$ be a simply-connected convex normal neighborhood.
If the operational data $\mathcal O_{\rm op}[U]$ satisfy the
Einstein--Bianchi system, then there exists a unique spacetime geometry
$(g_{\mu\nu},T_{\mu\nu})$ on $\mathcal U$, up to diffeomorphisms, that
reproduces all operational measurements.
\end{theorem}

\begin{proof}[Sketch]
Sections~\ref{eq:1p3}--\ref{eq:Delta} reconstruct
$\theta$ and $\sigma_{\mu\nu}$ from photon transport.
Section~3 reconstructs $\omega_{\mu\nu}$ from temporal holonomy.
Matter variables fix $R_{\mu\nu}$ via Einstein’s equation.
The Bianchi constraints \eqref{eq:E-constraint}--\eqref{eq:B-constraint}
and evolution equations determine $E_{\mu\nu}$ and $B_{\mu\nu}$, hence
the full curvature.  Cartan’s equations then integrate the orthonormal
frame and metric uniquely modulo diffeomorphisms.
\end{proof}

% =========================================================
\section{Examples}

\paragraph{Plane gravitational wave.}
For an observer congruence adapted to a plane gravitational wave,
shear measurements determine $\dot h_{+,\times}$ via the standard
relation between $\sigma_{\mu\nu}$ and the transverse-traceless metric
perturbation.  The Bianchi evolution then yields curvature components
$\propto \ddot h_{+,\times}$, and Cartan integration reconstructs the
local TT waveform up to gauge.

\paragraph{Weak static potential.}
For static observers in a weak-field metric,
$a_i=\partial_i\Phi$ directly measures the Newtonian potential’s gradient.
Einstein’s equation reduces to Poisson’s equation
$\nabla^2\Phi=4\pi\rho$.  Integrating recovers $\Phi$ and hence the
metric to the relevant order.

\paragraph{Rotating congruence.}
In flat spacetime, a rigidly rotating congruence has $\omega_{\mu\nu}\neq0$
but vanishing curvature.  Temporal holonomies detect non-zero vorticity
even though $R_{\mu\nu\rho\sigma}=0$, distinguishing rotation from gravity.

% =========================================================
\section{Discussion}

The key result is that all kinematical components of $\nabla_\mu U_\nu$
(shear, expansion, vorticity, and acceleration) are operationally
measurable: the symmetric part from photon frequency transport and the
antisymmetric part from temporal holonomy.  Together with matter
variables, the Einstein--Bianchi system determines the curvature, and
Cartan integrability reconstructs the metric.

Although $n^\mu$ enters intermediate constructions such as $\alpha$ and
the temporal connection, the reconstructed geometric quantities are
independent of this choice.  The temporal connection introduced in
Paper~I provides an operational route to vorticity and will play a
central role in the temporal gauge structure developed in subsequent
papers.

% =========================================================
\begin{thebibliography}{99}

\bibitem{PaperI}
S.~S.~Galagali,
``Operational Time Geometry in General Relativity,'' (2025).

% --- Core GR & Congruence Kinematics ---
\bibitem{Ellis1967}
G.~F.~R.~Ellis,
``Dynamics of pressure-free matter in general relativity,''
J.\ Math.\ Phys.\ \textbf{8}, 1171 (1967).

\bibitem{Ehlers1961}
J.~Ehlers,
``Contributions to the Relativistic Mechanics of Continuous Media,''
Gen.\ Rel.\ Grav.\ \textbf{25}, 1225 (1993)
(English translation of the 1961 original).

\bibitem{HawkingEllis}
S.~W.~Hawking and G.~F.~R.~Ellis,
\textit{The Large Scale Structure of Space-Time},
Cambridge University Press (1973).

\bibitem{EllisMaartensMacCallum}
G.~F.~R.~Ellis, R.~Maartens, and M.~A.~H.~MacCallum,
\textit{Relativistic Cosmology},
Cambridge University Press (2012).

% --- Cartan Geometry, Curvature Reconstruction ---
\bibitem{Cartan1922}
E.~Cartan,
``Sur les variétés à connexion affine,'' 
Ann.\ Éc.\ Norm.\ Sup.\ \textbf{39}, 325 (1922).

\bibitem{Sharpe}
R.~W.~Sharpe,
\textit{Differential Geometry: Cartan's Generalization of Klein's Erlangen Program},
Springer (1997).

\bibitem{Frankel}
T.~Frankel,
\textit{The Geometry of Physics},
Cambridge University Press (2011).

% --- Bianchi, Weyl, and 1+3 Curvature Decomposition ---
\bibitem{MaartensBianchi}
R.~Maartens and B.~A.~Bassett,
``Gravito-electromagnetism,''
Class.\ Quant.\ Grav.\ \textbf{15}, 705 (1998).

\bibitem{EllisVanElst}
G.~F.~R.~Ellis and H.~van Elst,
``Cosmological Models,''
in \textit{Theoretical and Observational Cosmology}, Springer (1999).

\bibitem{Friedrich}
H.~Friedrich,
``Hyperbolic reductions for Einstein's equations,''
Class.\ Quant.\ Grav.\ \textbf{13}, 1451 (1996).

% --- Operational Relativity, Relativistic Metrology, Clocks ---
\bibitem{Synge1960}
J.~L.~Synge,
\textit{Relativity: The General Theory},
North-Holland (1960).

\bibitem{Brumberg}
V.~A.~Brumberg,
\textit{Essential Relativistic Celestial Mechanics},
Adam Hilger (1991).

\bibitem{Kopeikin1999}
S.~M.~Kopeikin, M.~Efroimsky and G.~Kaplan,
\textit{Relativistic Celestial Mechanics of the Solar System},
Wiley (2011).

\bibitem{AshbyGPS}
N.~Ashby,
``Relativity in the Global Positioning System,''
Living Rev.\ Relativ.\ \textbf{6}, 1 (2003).

\bibitem{Delva2017}
P.~Delva and J.~Lodewyck,
``Atomic clocks: new prospects in metrology and geodesy,''
Acta Futura \textbf{10}, 67 (2017).

\bibitem{Wolf2016}
P.~Wolf et al.,
``Testing Relativity Using Space Clocks,''
Gen.\ Rel.\ Grav.\ \textbf{48}, 10 (2016).

% --- Shear, Vorticity, and Local Measurements ---
\bibitem{RamosShear}
A.~Ramos and C.~F.~Sopuerta,
``Gravitational-wave measurements of shear and expansion,''
Phys.\ Rev.\ D \textbf{100}, 104016 (2019).

\bibitem{SopuertaKinematics}
C.~F.~Sopuerta,
``Kinematical quantities in general relativity and their measurement,''
Class.\ Quant.\ Grav.\ \textbf{21}, 771 (2004).

% --- Raychaudhuri, Focusing, Causality ---
\bibitem{Raychaudhuri1955}
A.~Raychaudhuri,
``Relativistic Cosmology I,''
Phys.\ Rev.\ \textbf{98}, 1123 (1955).

\bibitem{KarSenGuptaRaychaudhuri}
S.~Kar and S.~SenGupta,
``The Raychaudhuri equations: A brief review,''  
Pramana \textbf{69}, 49 (2007).

% --- Standard Textbooks (Credibility Block) ---
\bibitem{Wald}
R.~M.~Wald,
\textit{General Relativity},
University of Chicago Press (1984).

\bibitem{MTW}
C.~W.~Misner, K.~S.~Thorne, and J.~A.~Wheeler,
\textit{Gravitation},
Freeman (1973).

\bibitem{PoissonToolkit}
E.~Poisson,
\textit{A Relativist's Toolkit},
Cambridge University Press (2004).

\bibitem{Choquet}
Y.~Choquet--Bruhat,
\textit{General Relativity and the Einstein Equations},
Oxford University Press (2009).

\end{thebibliography}


\end{document}
